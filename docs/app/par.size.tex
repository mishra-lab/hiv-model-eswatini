%===================================================================================================
\subsection{Activity Group Sizes}\label{mod.par.size}
We model population sizes of all activity groups as proportions of the total population,
which are assumed to remain roughly constant.
Individuals can, however, move between groups (see \sref{mod.par.turn.act})
--- \ie groups are open populations ---
and disproportionate mortality due to HIV between groups
may cause higher risk groups to shrink over time.
Overall population growth is discussed in \sref{mod.par.turn.bd}.
%---------------------------------------------------------------------------------------------------
\subsubsection{Female Sex Workers}\label{mod.par.size.fsw}
The proportion of women who report sex work in national demographic and health surveys
is generally considered unreliable due to social desirability bias,
particularly if the survey is face-to-face and household-based
\cite{Konings1995,Gregson2002bias,Gregson2004,Lowndes2012,Behanzin2013}.
Therefore, FSW population size estimates require
targeted surveys and unique methodologies \cite{UNAIDS2010kps,Abdul-Quader2014}.
In both \cite{EswKP2014} and \cite{EswIBBS2022}, the Swati FSW population size
was estimated using a combination of
unique object method, service multiplier method, prior survey participation,
and network scale-up method (NSUM) \cite{UNAIDS2010kps}.
In 2011 \cite{EswKP2014}, regional FSW population size estimates
ranged from 0.7\% to 6.5\% of all women,
with overall population-weighted mean across regions of 2.9\%;
in 2021 \cite{EswIBBS2022}, the mean (95\%~CI) estimates were 2.43~(1.17,~5.02)\%.
To reflect this uncertainty in the model, we fit a BAB distribution
such that 95\% of the probability fell between 0.7\% and 6.5\%,
and used as the prior distribution for the proportion of women who are FSW.
Then, following the analysis in \sref{mod.par.fsw},
we fixed the proportion of all FSW in the higher risk FSW group at 20\%,
and likewise the lower risk group at 80\%.
%---------------------------------------------------------------------------------------------------
\subsubsection{Clients of FSW}\label{mod.par.size.cli}
Similar to FSW, household-based surveys are not considered reliable data sources
for estimating the population size of clients of FSW \cite{Behanzin2013}.
However, few surveys are designed to reach clients of FSW,
and no direct estimates of FSW size exist for Eswatini.
So, we use a common approach for inferring the FSW client size \cite{Cote2004},
similar to the ``multiplier method'' \cite{Morison2001}.
Given the FSW population proportion $P_{\txc{fsw}}$,
the average number of yearly new and regular sex work clients per FSW $\bar{Q}_{\,\txc{fsw}}$,
the frequency of sex per partnership-year $F_{\txc{sw}}$, and
the average total number of yearly sex acts per client
$\bar{Q}_{\,\txc{Cli}} F_{\txc{sw}}$,
we define the total client population $P_{\txc{cli}}$ as:
\begin{equation}
  P_{\txc{cli}} = \frac{P_{\txc{fsw}}\,\bar{Q}_{\,\txc{fsw}}\,F_{\txc{sw}}}
                                            {\bar{Q}_{\,\txc{cli}}\,F_{\txc{sw}}}
\end{equation}
Then, as with FSW, the proportion of total clients in the higher risk client group
is defined as 20\% of all clients, and likewise for the lower risk group at 80\%.
Using $\bar{Q}_{\,\txc{fsw}}$, $\bar{Q}_{\,\txc{cli}}$, and $F_{\txc{sw}}$
as defined below in \sref{mod.par.pnum.sw}, the prior client population size $P_{\txc{cli}}$
estimated by this method was 11.6~(2.1,~34.4)\% of men.
%---------------------------------------------------------------------------------------------------
\subsubsection{Wider Population}\label{mod.par.size.wp}
Based on the results of \sref{mod.par.wp},
we defined the sizes of the modelled lower and medium activity groups,
and the average numbers of main/spousal partnerships per person.
We assumed that $W'_{2+}$ and $M'_{2+}$ included FSW and client population sizes, respectively.
Thus, we defined the populations size of medium activity women as
$W_{\txc{m}} = W'_{2+} - W_{\txc{fsw}}$.
Sampling $W'_{2+}$ from a BAB distribution with 95\%~CI (10,~27)\%,
the resulting 95\%~CI for medium activity women $W_{\txc{m}}$ was (6,~25)\% of women.
We then defined the lowest activity women population size as $W_{\txc{l}} = 1 - W'_{2+}$,
representing (72,~90)\% of women.
Since there is greater uncertainty in the client population size,
the same approach for the medium activity men population size could yield negative values.
Instead, we sampled the proportion of medium activity men $M_{\txc{m}}$ directly from
a BAB distribution with 95\%~CI (10,~17)\%, yielding
95\%~CI for $M_{\txc{m}} + M_{\txc{cli}}$ of (14,~52)\% of men,
which is close to (15,~44)\% from $M_{2+}$.
We then defined the lowest activity men were as
$M_{\txc{l}} = 1 - M_{\txc{m}} + M_{\txc{cli}}$,
representing (48,~86)\% of men.
