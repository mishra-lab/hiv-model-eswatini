%===================================================================================================
\subsection{Turnover}\label{mod.par.turn}
%---------------------------------------------------------------------------------------------------
\subsubsection{Births \& Deaths}\label{mod.par.turn.bd}
The modelled population considers ages 15--49,
reflecting commonly reported data and the majority of sexual activity.
In the absence of mortality, individuals would therefore
remain within the modelled ``open cohort'' population for 35 years.
The estimated average yearly mortality rate for these ages was 1.44\% around 2006
\cite[Table~15.2]{SDHS2006}.
However, this estimate includes HIV/AIDS-attributable mortality,
which we model separately (see \sref{mod.par.hiv.mort}),
accounting for approximately 64\% of deaths around that time \cite{WHO2006esw}.
Thus, the overall exit rate from the modelled cohort
due to reaching age 50 (``aging out'') and non-HIV-attributable mortality was:
$\mu = 1/35 + (1-.64) 1.44\% = 3.78\%$.
\par
We estimated the rate of entry into the modelled population $\nu$
to fit population size of ages 15--49 in Eswatini \cite{WorldBank},
and approximate population growth rates \cite{UNWPP2019},
given that we model HIV/AIDS-attributable mortality separately.
Specifically, we assumed a population growth rate $g = \nu - \mu$ in the absence of HIV/AIDS of
4\% in 1980, 3\% in 2000, 1.5\% in 2010, and 1.5\% in 2020 (monotonic cubic interpolation).
We sampled $g$ in 2050 from a uniform prior with 95\% CI (0.7\%,~1.5\%),
reflecting uncertainty in estimated projections \cite{UNWPP2019}.
Finally, we calculated the population entry rate as $\nu = g + \mu$.
These parameter values were informally validated by comparison of model outputs with
Swati population sizes for ages 15--49 from \cite{WorldBank}.
The distribution of activity groups among individuals \emph{entering} the model, denoted $E_{si}$,
is different from the distribution among individuals \emph{currently} in the model $P_{si}$,
but $E_{si}$ is computed automatically as described below in \sref{mod.par.turn.act}.
%---------------------------------------------------------------------------------------------------
\subsubsection{Activity Group Turnover}\label{mod.par.turn.act}
In addition to overall population turnover (entry/exit from the open population),
we model movement of individuals between activity groups within the model.
Activity group turnover reflects the fact that risk is not constant over sexual life course,
and reported duration in higher activity contexts can be short \cite{Scorgie2012}.
Previous modelling has shown that activity group turnover (sometimes called ``episodic risk'')
can strongly influence parameter fitting and intervention impact \cite{Henry2015,Knight2020}.
We model turnover from activity group $si$ to $si'$ as a constant rate $\theta_{sii'}$,
which implies an assumption that (in the absence of HIV) duration in group $si$ is
exponentially distributed with mean $D_{si}$ \cite{Roberts2015}:
\begin{equation}\label{eq:model.par.dur}
  D_{si} = \frac{1}{\mu + \sum_{i'}\theta_{sii'}}
\end{equation}
where $\mu$ is the overall exit rate from \sref{mod.par.turn.bd}.
As shown previously \cite{Knight2020}, the relative sizes of each sex-activity group $P_{si}$
can be maintained at fixed values by satisfying the following ``mass-balance'' equation:
\begin{equation}
  \nu P_{si} = \nu E_{si} + \sum_{i'} \theta_{si'i} P_{si'} - \sum_{i} \theta_{sii'} P_{si}
\end{equation}
Specific turnover rates $\theta_{sii'}$ and entrant activity group sizes $E_{si}$
can then be uniquely resolved by specifying
$N_i\,(N_i-1) = 12$ non-redundant and compatible constraints,
where specifying each $D_{si}$ is one such constraint.
% ---------------------------------------------------------------------------------------------------
\paragraph{Selling Sex}
Estimating durations (\eg in sex work) from cross-sectional data
should consider several potential sources of bias \cite{Fazito2012,Knight2023bias},
including distributional, sampling, censoring, and measurement biases.
We previously explored these biases using the 2011 Eswatini FSW survey data \cite{Baral2014}
and inferred adjusted estimates of sex work duration
via a Bayesian Bayesian hierarchical model \cite{Knight2023bias}.
We estimated a mean duration of 4.06~(2.29,~6.34) years,
with durations distributed approximately exponentially
--- compatible with the implicit assumption of compartmental models \cite{Anderson1991}.
Thus, we sampled overall duration in sex work from a gamma prior with 95\%~CI (2.29,~6.34) years.
As noted in \sref{mod.par.fsw}, we conceptualized higher risk sex work as
a transient period, with short duration.
We sampled this duration from a gamma prior with 95\%~CI (2,~12) months.
We then modelled turnover for higher risk sex work as
exclusively coming from / going to lower risk sex work,
including no direct entry from outside the model: $E_{si} = 0$.
By contrast, women entering lower risk sex work could
enter directly from outside the model ($E_{si} > 0$),
and turnover from / to any other activity group.
%---------------------------------------------------------------------------------------------------
\paragraph{Buying Sex}
Data to inform the average duration spent buying sex among clients is limited.
\citet{Fazito2012} estimated mean durations of 4.6--5.5 years
based on studies in Benin \cite{Lowndes2000} and Kenya \cite{Voeten2002}.
\citet[Table~G]{Hodgins2022} also gives pooled estimates for
the proportions of men in Sub-Saharan Africa
who paid for sex \emph{ever} \vs in \emph{p12m} during 2000--2020.
Estimates ranged from 8.8~(6.5,~11.7)\% of men aged 25--34 who ever bought sex,
to 2.2~(1.5,~3.2)\% of men aged 35--54 who bought sex in p12m.
Based on these data, we defined a gamma prior distribution for
the total duration buying sex with 95\%~CI (4,~15) years.
We conceptualized higher risk clients as a transient period
with the same duration as higher risk sex work,
and assumed an equal pattern of possible turnover between activity groups
among men buying sex as women selling sex.
%---------------------------------------------------------------------------------------------------
\paragraph{Lowest \& Medium Activity Groups}
Data on individual-level changes to numbers of non-sex work partners in p12m
is even more sparse than data related to sex work;
so, it's unclear to what extent individuals move between the lowest and medium activity groups
throughout their sexual life course.
Data from Uganda, Zimbabwe, and South Africa \cite{Todd2009}
suggested that sexual activity (proportion sexually active and mean numbers of partners)
was approximately stable with age (after sexual debut and and before age 49),
with modest trends toward lower activity at older age.
However, these population-level data do not necessarily suggest that
the \emph{same} individuals have multiple partnerships each year.
Reflecting this uncertainty, we sampled
the rate of turnover from medium to lowest activity for both women and men
from a gamma prior with 95\%~CI (5,~50)\% per year.
%---------------------------------------------------------------------------------------------------
\paragraph{Additional Turnover Assumptions}
The above assumptions specify 8 constraints for each sex:
2 durations $D_{si}$, 1 entry rate $E_{si} = 0$, and
5 turnover rates $\theta_{sii'}$ (4 zero, 1 nonzero).
Next, since FSW often enter sex work shortly after sexual debut \cite{Cheuk2020,Ma2020},
and sexual activity is roughly constant or slightly declining with age \cite{Todd2009},
we assumed that $E_{si} = f\,P_{si}$,%
\footnote{Subject to $f \le (\nu - \mu + D_{si}^{-1})\,\nu^{-1}$,
  which can be derived from Eq.~(10) in \cite{Knight2020}.}
with $f = 2$ for lower risk FSW, $f = 1.5$ for lower risk clients,
and $f = 1$ for medium activity women and men (+2~constraints);
then $f < 1$ for the lowest activity women and men is computed automatically.
Finally, since exiting sex work is unlikely to be
an abrupt transition to monogamous or zero sexual activity \cite{Scorgie2012,Learmonth2015},
we further assumed that (50,~90)\% of women exiting sex work
transition to the medium activity group (BAB prior) (+1~constraint);
in the absence of relevant data, we made a similar assumption regarding clients,
with (25,~90)\% former clients transitioning to the medium activity group (+1~constraint).
These $10 < 12$ total constraints then allow two degrees of freedom to resolve
the values of $\theta_{sii'}$ and $E_{si}$.
A non-negative solution to the system of constraints is solved as described in \cite{Knight2020},%
\footnote{Using \hreftt{docs.scipy.org/doc/scipy/reference/generated/scipy.optimize.nnls.html}}
repeated at each timestep since $\nu$ varies with time.
