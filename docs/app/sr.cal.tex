\section{Model Calibration}\label{sr.cal}
This section presents the results of model calibration
under the \emph{Effective Partnerships Adjustment}, including:
\sref{sr.cal.par}: posterior distributions of calibrated parameters (Table~\ref{tab:par.cal});
\sref{sr.cal.targ}: model outputs \vs associated calibration targets; and
\sref{sr.cal.wiw} modelled patterns of transmission among risk groups in Eswatini over time.
%===================================================================================================
\subsection{Posterior Parameter Distributions}\label{sr.cal.par}
Figure~\ref{fig:post.distr} illustrates the distributions of calibrated model parameters,
stratified by prior (IMIS iteration 0) \vs posterior (resampled 1000 parameter sets).
Most prior \vs posterior distributions were significantly different
(Anderson-Darling Test \cite{Scholz1987}).
Such differences tended to favour increased overall transmission
(\eg \texttt{beta_0}, \texttt{dur_acute}, \texttt{Rbeta_acute}),
including via casual partnerships
(\eg \texttt{C12m_cas_wm}, \texttt{RC_cas_cli.wm}, \texttt{RF_cas.msp}),
but decreased accumulation of HIV prevalence among FSW
(\eg \texttt{F_swr}, \texttt{dur_fsw}).
These findings may reflect the difficulty of recreating high overall HIV prevalence
despite relatively small prevalence ratios (\ttilde 2) for FSW \vs women overall.
Relative transmission from men to women (\texttt{Rbeta_vi_rec}) was also high,
perhaps because our model lacks age, and
age-gender differences in sexual activity and/or mixing might better explain
gender differences in HIV prevalence.
\begin{figure}
  \centering\includegraphics[width=\linewidth]{post.distr.stats}
  \caption{Distributions of calibrated model parameters,
    stratified by prior (all initial samples) \vs posterior (1000 resamples)}
  \label{fig:post.distr}
  \floatfoot{\ffpar; \ffpost;
    \ffast{QN rank score test \cite{Kruskal1952} for comparing distributions}}
\end{figure}
\par
Figure~\ref{fig:post.cor} further illustrates
bivariate rank correlations among posterior parameter values.
Several sex work condom use parameters had strong positive correlations,
likely induced via relational constraints (see \sref{mod.cal.appr}).
By contrast, the strongest negative correlation was between
the rate of diagnosis for non-FSW women and a global relative rate for diagnosis,
since these parameters served very similar roles.
\begin{figure}
  \centering\includegraphics[width=\linewidth]{post.cor}
  \caption{Rank correlations among posterior samples for calibrated model parameters}
  \label{fig:post.cor}
  \floatfoot{\ffpar.}
\end{figure}
%===================================================================================================
\subsection{Model Fits \vs Calibration Targets}\label{sr.cal.targ}
This section presents the estimates of key model outputs from the \fffit,
with comparison to the associated calibration targets.
%---------------------------------------------------------------------------------------------------
\paragraph{Log Likelihoods}
Figure~\ref{fig:ll.hist} illustrates the distributions of log likelihoods for
initial prior / Latin hypercube samples (100,000),
all IMIS iteration samples (1,000,000),
and posterior samples (1000).
See \sref{mod.cal.appr} regarding our adjusted IMIS weights
due to the large variance in log likelihoods.
\begin{figure}
  \centering\includegraphics[scale=1]{ll.hist}
  \caption{Distribution of log likelihoods across stages of calibration}
  \label{fig:ll.hist}
  \floatfoot{Only log likelihoods $\ge -1000$ shown.}
\end{figure}
\pagebreak % TEMP
%---------------------------------------------------------------------------------------------------
\subsubsection{HIV Prevalence \& Incidence}\label{sr.cal.hiv}
Figure~\ref{fig:fit.hiv} illustrates modelled HIV
prevalence \sfref{fig:fit.prev},
prevalence ratios \sfref{fig:fit.prev1v2},
incidence \sfref{fig:fit.inc}, and
incidence ratios \sfref{fig:fit.inc1v2} among selected risk groups,
with the associated calibration targets.
Overall, model estimates agree well with the available calibration targets,
with the following shortcomings.
Relative to the calibration targets, the model tends to
overestimate HIV prevalence among FSW in 2014,
slightly overestimate HIV prevalence among men overall, and
overestimate HIV prevalence ratios for non-lowest \vs lowest activity men.
The discrepancies for men could arise due to population-level surveys
failing to reach men at higher risk (\eg with high mobility) \cite{Camlin2016},
considering that participation rates were consistently lower for men \vs women
(\sref{mod.par.data}).
\begin{figure}
  \subcapoverlap
  \foreach \var/\fw in {prev/1,prev1v2/1,inc/1,inc1v2/.67,prevanc/.33}{
    \begin{subfigure}{\fw\linewidth}
      \centering\includegraphics[scale=\fitscale]{fit.\var.base.all}
      \caption{\raggedright}
      \label{fig:fit.\var}
    \end{subfigure}}
  \caption{Modelled HIV prevalence, incidence, and ratios thereof among selected risk groups,
    and associated calibration targets}
  \label{fig:fit.hiv}
  \floatfoot{\fffit; \ffribbon; \ffpbar;
    \sfref{fig:fit.prevanc} targets from antenatal care data \cite{NERCHA2012rep,EswMOH2015rep}
    (not used for calibration)}
\end{figure}
\par
Few data are available to validate the modelled early epidemic dynamics.
Modelled incidence among women and men peaked rapidly after introduction of HIV
(Figure~\ref{fig:fit.inc}), corresponding to
rapid acquisition and saturation among FSW and clients.
Modelled incidence and prevalence continued to increase approximately linearly over 1990--2010,
reflecting a balance of would-be exponential epidemic growth
\vs build-up of mitigating factors, such as
increasing condom use (Figure~\ref{fig:fit.condom}),
male circumcision (Figure~\ref{fig:fit.circum}),
ART coverage (Figure~\ref{fig:fit.cascade}, and
accumulation of seroconcordant partnerships (Figure~\ref{fig:fit.tdsc}).
These trends can be compared with HIV prevalence from Eswatini antenatal care clinics
over the same period (Figure~\ref{fig:fit.prevanc}), which suggest similar trends.%
\footnote{Antenatal care data were not used as calibration targets because
  such data are known to overestimate HIV prevalence among women overall
  due to non-representative sampling \cite{Gouws2008,Marsh2014}.}
Decline of HIV incidence and prevalence after 2010 can likely be attributed to
rapid ART scale-up (see \sref{sr.cal.cascade})
and further increases in condom use (Figure~\ref{fig:fit.condom}).
Although modelled incidence declined rapidly, prevalence remained relatively higher
due to increased survival of PLHIV with ART.
In some model fits, prevalence among FSW declined faster than among women overall,
likely due to high turnover of women in sex work.
%---------------------------------------------------------------------------------------------------
\subsubsection{ART Cascade}\label{sr.cal.cascade}
Figure~\ref{fig:fit.cascade} illustrates the modelled ART cascade among selected risk groups,
including both conditional and unconditional cascade steps,
and the associated calibration targets.
The model estimates agree quite well with these targets, for all risk groups.
The non-monotonic increasing proportions virally suppressed among treated PLHIV
reflect major changes in treatment eligibility (see \sref{mod.par.cascade.tx}), which caused
influxes of newly ART-eligible PLHIV to
temporarily decrease the proportions virally suppressed among treated PLHIV.
Figure~\ref{fig:fit.rates} also illustrates
rate of HIV diagnosis \sfref{fig:fit.dx.rate} and ART initiation \sfref{fig:fit.tx.rate}
among selected risk groups over time.
\begin{figure}
  \subcapoverlap
  \foreach \var in {diag,treat.c,treat.u,vls.c,vls.u}{
  \begin{subfigure}{\linewidth}
    \centering\includegraphics[scale=\fitscale]{fit.\var.base.all}
    \caption{\raggedright}
    \label{fig:fit.\var}
  \end{subfigure}}
  \caption{Modelled ART cascade among selected risk groups
    and associated calibration targets}
  \label{fig:fit.cascade}
  \floatfoot{\ffcas; \fffit; \ffribbon; \ffpbar.}
\end{figure}
\begin{figure}
  \subcapoverlap
  \foreach \var in {dx,tx}{
    \begin{subfigure}{\linewidth}
      \centering\includegraphics[scale=\fitscale]{fit.\var.rate.base.all}
      \caption{\raggedright}
      \label{fig:fit.\var.rate}
    \end{subfigure}}
  \caption{Modelled rates of HIV diagnosis and ART initiation among selected risk groups over time}
  \label{fig:fit.rates}
  \floatfoot{\fffit; \ffribbon.}
\end{figure}
%---------------------------------------------------------------------------------------------------
\subsubsection{Additional Model Outputs}\label{sr.cal.other}
Figure~\ref{fig:fit.Psi} illustrates
the relative sizes of higher and lower risk FSW and clients over time,
showing relative stability despite disproportionate HIV-attributable mortality.
Figure~\ref{fig:fit.NX} illustrates the total population size over time,
showing agreement with available data \cite{WorldBank} (see \sref{mod.par.turn.bd}).
Figure~\ref{fig:fit.circum} illustrates
the modelled proportion of men aged 15--49 who are circumcised,
including uncertainty about future trends (see \sref{mod.par.tm.circ}).
Figure~\ref{fig:fit.condom} illustrates
modelled trends in condom use within different partnership types (see \sref{mod.par.tm.condom}).
Figure~\ref{fig:fit.tdsc} illustrates
the modelled transmission-driven seroconcordance proportion
for different partnership types (see \sref{foi.prop.tdsc}),
as defined in \eqref{eq:tdsc} with denominator (b).
\begin{figure}
  \centering\includegraphics[scale=\fitscale]{fit.Psi.base.all}
  \caption{Modelled relative sizes of selected risk groups over time}
  \label{fig:fit.Psi}
  \floatfoot{\fffit; \ffribbon.}
\end{figure}
\begin{figure}
  \begin{minipage}[t]{.45\linewidth}
    \centering\includegraphics[scale=\fitscale]{fit.NX.base.all}
    \caption{Modelled total population aged 15--49
      and associated calibration targets}
    \label{fig:fit.NX}
  \end{minipage}\hfill
  \begin{minipage}[t]{.45\linewidth}
    \centering\includegraphics[scale=\fitscale]{fit.circum.base.all}
    \caption{Modelled proportion of men aged 15--49 who are circumcised}
    \label{fig:fit.circum}
  \end{minipage}
  \floatfoot{\fffit; \ffribbon; \ffpbar.}
\end{figure}
\begin{figure}
  \centering\includegraphics[scale=\fitscale]{fit.condom.base.all}
  \caption{Modelled condom use within different partnership types}
  \label{fig:fit.condom}
  \floatfoot{\fffit; \ffribbon.}
\end{figure}
\begin{figure}
  \centering\includegraphics[scale=\fitscale]{fit.tdsc.base.all}
  \caption{Modelled transmission-driven seroconcordance within different partnership types}
  \label{fig:fit.tdsc}
  \floatfoot{\fffit; \ffribbon.}
\end{figure}
%===================================================================================================
\subsection{Who Infected Whom}\label{sr.cal.wiw}
As further model validation, and to gain insights into the modelled networks of transmission,
this section presents several summaries of ``who infected whom''
--- \ie distributions of yearly infections stratified by
the transmitting group, acquiring group, and partnership type.
Throughout the section, the numbers of yearly infections shown are obtained from
the median value across all 1000 model fits.
\par
Figure~\ref{fig:wiw.base.frto} illustrates
the total numbers and proportions of modelled yearly infections
transmitted from \sfref{fig:wiw.base.from} and
acquired among \sfref{fig:wiw.base.to} modelled risk groups.
Figure~\ref{fig:wiw.base.ratio} then gives
the \emph{ratio} of yearly infections transmitted \vs acquired.
Figure~\ref{fig:wiw.base.part} stratifies yearly infections by partnership type, while
Figure~\ref{fig:wiw.base.alluvial} illustrates
the complete transmission network every 10 years from 1990.
\begin{figure}
  \begin{subfigure}{.5\linewidth}
    \includegraphics[width=\linewidth]{wiw.base.from}
    \caption{Transmitted from}
    \label{fig:wiw.base.from}
  \end{subfigure}%
  \begin{subfigure}{.5\linewidth}
    \includegraphics[width=\linewidth]{wiw.base.to}
    \caption{Acquired among}
    \label{fig:wiw.base.to}
  \end{subfigure}
  \caption{Modelled yearly HIV infections
    \sfref{fig:wiw.base.from} transmitted from and \sfref{fig:wiw.base.to} acquired among risk groups}
  \label{fig:wiw.base.frto}
  \floatfoot{\ffpops; \ffwiw.}
\end{figure}
\begin{figure}
  \begin{minipage}[b]{.49\linewidth}
    \includegraphics[width=\linewidth]{wiw.base.ratio}
    \caption{Ratio of modelled yearly infections transmitted from \vs acquired among risk groups}
    \label{fig:wiw.base.ratio}
  \end{minipage}\hfill
  \begin{minipage}[b]{.49\linewidth}
    \includegraphics[width=\linewidth]{wiw.base.part}
    \caption{Modelled yearly HIV infections transmitted via different partnership types}
    \label{fig:wiw.base.part}
  \end{minipage}
  \floatfoot{\ffpops; \ffwiw.}
\end{figure}
\begin{figure}
  \includegraphics[width=\linewidth]{wiw.base.alluvial}
  \caption{Alluvial diagram showing proportions of all yearly infections (ribbons)
    transmitted from (left) to (right) modelled risk groups,
    stratified by partnership type (color) and year (facets)}
  \label{fig:wiw.base.alluvial}
  \floatfoot{\ffpops; \ffwiw.}
\end{figure}