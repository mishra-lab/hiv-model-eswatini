\section{Purpose}
For decades, standard compartmental models of sexually transmitted infections
have simulated sexual partnerships as ``instantaneous''.
However, this instananeous approach can bias model dynamics and outputs, such as
underestimating the impact of preventing transmission in sex work.
We developed a new approach to simulating sexual partnerships in compartmental models
which moves beyond the instananeous paradigm, and avoids these biases.
\section{Methods}
In the instananeous approach, people change partners at least once per year,
or faster for shorter and/or simultaneous partnerships per-person.
Then, a fraction of people changing partners become infected,
reflecting the cumulative probability of transmission per-partnership/year (Figure~a,~left).
In the proposed approach, we define a rate of transmission per-partnership, reflecting sex frequency;
such rates can be summed across simultaneous partnerships per-person.
Then, we track the number of people who recently transmitted or acquired infection;
we decrease by one the effective numbers of partners among these people,
until they form a new partnership, determined by partnership duration (Figure~a,~right).
We integrated the proposed approach, and the instananeous approach, separately,
into a model of heterosexual HIV transmission in Eswatini.
After calibrating the model under the proposed approach,
we compared modelled HIV incidence under the proposed versus the instananeous approach,
with the same model parameters.
\section{Results}
% SM: past tense or present?
% JK: The dual tenses was originally intended, and I still kind of prefer it -- what do you think?
%     - past tense to describe what we observed in the incidence plot,
%     - but present tense to describe the mechanistic descriptions,
%       as these are generalizable insights that are "true then and now and forever"
%     Something about writing the second part in past tense sounds kinda wrong to me
Incidence under the instananeous approach was
consistently greater than under the proposed approach,
with relative differences growing over time (Figure~b).
These differences can be explained mechanistically as follows.
As prevalence increases, both approaches capture population-level herd effects:
reduced onward transmission from each infection due to
\emph{new} parnterships forming between two infected people (assumed to be at random).
However, the proposed approach also captures partnership-level herd effects:
reduced transmission due to \emph{existing} partnerships
continuing between two infected people following transmission.
These continuing partnerships cannot be modelled in the instananeous approach,
and so infected individuals are assumed to be immediately at risk of onward transmission.
Thus, as partnership-level herd effects accumulate over time,
relative differences in incidence under the instananeous versus the proposed approach
also grow over time.
\section{Conclusions}
Modelling sexual partnerships as instananeous can cause
compartmental models of HIV transmission to overestimate HIV incidence,
especially in mature and declining epidemics.
The proposed approach offers a generalizable solution to move beyond instantaneous partnerships
in compartmental models of sexually transmitted infections,
and captures key epidemic dynamics related to partnership-level herd effects.
