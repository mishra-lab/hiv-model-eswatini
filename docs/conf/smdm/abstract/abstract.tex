\section{Purpose} %SM: great abstract! tough to convey easily and edits suggested largely to make it easier for reader (who may not be super super familiar with STI modeling). minor point == lots of "this" used and its a bit hard to follow what "this" refers to at times :) 
Classic compartmental models of sexually transmitted infections
simulate sexual partnerships as ``instantaneous'' by
applying the cumulative risk of transmission per-partnership
to the fraction of people forming new partnerships per unit time. 
Thus, the usual partnership model does not account for partnership duration, %SM: usual or traditioanl and same term throughout
and could lead to biased estimates of the contribution of
long-term partnerships to overall transmission.
We developed a new partnership model to address this methodological gap in compartmental models.
\section{Methods}
In the proposed model, we parameterize different partnerships types using:
number of concurrent partnerships per-person, frequency of sex per-partnership, partnership duration;
versus the ``instantaneous'' approach: rate of partnership change, and total sex per-partnership. %SM: do we mean total sex acts?
For each type of partnership, we introduce a new strata to track individuals %SM: I know strictly its proporotions, but might be eaiser re: language/clarity to say people?
who recently transmitted or acquired infection via the partnership type and 
have not yet formed a new partnership of the same type.
Then, in the incidence equation for each partnersthip type, we reduce by one
the effective numbers of partnerships of among those who recently acquired/transmitted. %SM: define "those people"
We compared the influence of the``instantaneous' partnership model vs the proposed model by using each 
approach separately in an existing, data-driven model of heterosexual HIV transmission in Eswatini. 
We fit the Eswatini model using the proposed approach, 
and used the fitted paramters to simulate the HIV epidemic (overall HIV incidence) with the ``instantaneous''  approach.

\section{Results}  %SM: past tense or present? 
At the start of the epidemic, HIV incidence increased earlier and faster with the proposed model, but
then slowed and peaked at a lower magnitude, as compared with the ``instantaneous'  model (Figure). 
% this finding can be explained as follows.
The difference can be explained mechanistically as follows.
Early in the epidemic, the transmission potential is lower in the instantaneous model
because the \emph{probability} of transmission per-partnership-year in the instantaneous model
is less than the \emph{rate} of transmission in the proposed model,
due to survival effects (HIV transmission can only occur once per-partnership). %SM: am confused by the full sentence. "due to survival effects" applies to which model? suggest rephrase for clarity

Over time there is an accrual of individuals who already transmitted to their current partner. 
In the proposed model, such individuals are tracked and thus cannot pass on infections unless they explicitly form new parnterships, whereas 
the instantaneous model has no such restriction. Thus, the potential for onward transmission at a population-level declines faster in the proposed model, 
leading to lower HIV incidence after the growth phase of the epidemic.

%SM: not entirely true as currently written, so tried to edit a bit. i.e. still have herd effects with instantansouse model. so the latter part of the phrase might be misleading b/c we do still see the peak and fall due to natural dynamics too in an instantaneous partnership model.

\section{Conclusions}
Asusmptions of instananeous partnerships could lead compartmental models of HIV transmission to
underestimate early epidemic dynamics, and overestimate
HIV incidence in mature and declining HIV epidemics.
The proposed model offers a methodological solution to move beyond ``instantaneous'' partnerships
within compartmental models of sexually transmitted infections by
capturing key epidemic dynamics related to partnership duration without having to use an individual-based or pair-based model.
% JK: I mention "concurrency" in the intro & conclusion,
%     as the model really does capture these dynamics,
%     but clearly don't really get into those details due to the space,
%     plus, I know there are stigma issues potentially related to that word,
%     so maybe just better to leave it out? SM: agree with leaviing out the term concurrency 

%SM: figure - I don't understand the 2nd panel re: what "differnce vs. proposed" means. Could quite follow what the 2nd panel was trying to tell me...?