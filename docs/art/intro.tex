\section{Introduction}\label{art.intro}
% SM: great introduction section structure! main suggestions:
%     introduce term inequality; introduce the idea of overall vs inequality in 95-95-95
% JK: thanks! I've incorporated inequality more, but it is still a bit awkward
%     comparing the focus of abstract on cascade vs 1st intro para on ART.
%     Not sure how else to structure though.
% SM: I think we can flip and start with inequalities and then rest as you have it.
%     lets chat about it on Thursday - will ping you
%     (there are several openings, and will be eaisier to walk through together and then suggest edits).
% JK: Have revised the introduction a lot based on our chat.
%     I did find it hard to include inequalities in the first para though,
%     as there were already a lot of ideas to introduce there.
%     Instead, I focus the whole next para on inequalities, and then the last para is modelling.
%     So, the result is not exactly like we discussed, but I do think flows better than before
%     and focuses inequalities more centrally --- what do you think?
Early HIV treatment via antiretroviral therapy (ART) is a lifesaving intervention
increasing both the quantity and quality of life \cite{Lundgren2015init}.
A secondary benefit of early ART given Undetectable = Untransmittable (U=U) is that
% HM: Might need to spell out.
% JK: done
transmission risks are mitigated in serodifferent partnerships \cite{Cohen2016}.
To realize these benefits, massive efforts are underway to achieve
the UNAIDS 95-95-95 ART cascade targets \cite{UNAIDS2023} --- \ie to have:
% JK: do we need to spell out the meaning of 'cascade' -- i.e. steps?
% JK: done
95\% diagnosed among people living with HIV,
95\% on ART among those diagnosed, and
95\% virally suppressed among those on ART.
Botswana, Eswatini, Rwanda, Tanzania, and Zimbabwe
have already surpassed 95-95-95 nationally \cite{UNAIDS2023}, and
and achieving these targets is expected to help reduce HIV incidence towards elimination.
% SM: but did the real-world getting to 95-95-95 achieve the incidence reductions expected?
%     so far 5 countries have achieved (Estwani, Botswana, etc.).
%     I think we could consider not talking about the trials here,
%     but just say that inequalities exist and have not been fully examined re: their implications.
% SB: Important to really lay out the clear differences and disconnect between
%     individual observed benefits and population-level ones.  Ie, U+U is a fact.
%     So it is really that it didn't lower HIV incidence in the trials
%     that was something that necessitated interrogation.
% LW: Would be more explicit about the different findings from those trials vs. previous modeling studies.
% JK: Based on chat with SM, we no longer mention the trials,
%     but focus on countries achieving 95-95-95,
%     and then the potential influence of inequalities in the next para below
\par
Yet, there are growing concerns that inequalities in the ART cascade
could undermine the population-level prevention impacts of ART
anticipated from individual-level and model-based studies
\cite{Baral2019,Green2020,Maheu-Giroux2024}.
% SB: And individual-level studies.
% SM: maybe we introduce the term inequalities here?
% JK: done & done
% LW: Echo Stef's point of consistency: ART cascade vs. HIV treatment cascade vs. ART scale-up
% JK: have checked & tried to ensure consistency throughout the paper
Specifically, available data suggest that cascade attainment is often lower
among subpopulations at greater risk of HIV acquisition and/or transmission,
% SM: risk of?
% JK: acquisition and/or transmission
including key populations, younger men and women, and highly mobile populations
\cite{Hakim2018,Green2020}.
These inequalities can be driven by
systemic barriers to engagement in care faced by marginalized populations,
% SM: i've been learning from work in the social epidemiology field not to call the barriers unique,
%     but rather systemic (to reflect the world in whicih we live in re: systems and structures shaping barriers)...
%     i.e. they are unique b/c discrimination and "othering" is created by systems...
% JK: ah, good to know -- thanks! I've replaced 'unique' -> 'systemic' throughout (I think SB flagged too)
which intersect with individual, network, and structural determinants of HIV risk,
such as economic insecurity, mobility, stigma, discrimination, and criminalization
\cite{Wanyenze2016,Schwartz2017,Schmidt-Sane2022}.
% SM: what do we mean by drivers here? HIV risk of acquistion or onward transmission or both?
% RK: Perhaps a slight expansion would be useful, outlining some well-described barriers that FSW face in accessing care?
%     Words may be too short though I guess....
% SB: I would be clear about individual, network, and structural determinants really driving this.
% JK: Have added a list of specific factors
Moreover, cascade data may be lacking entirely
for subpopulations experiencing the greatest barriers to care
--- \ie the lowest ART cascades likely remain unmeasured \cite{Hakim2018}.
% SM: not 100% sure what we want to say here? agree, would clarify ...
%     e.g. do we mean something like: "However, current data systems for monitoring the HIV cascade
%     rarely do so by populations at differential risks of HIV transmission.
%     Thus, inequalities in the HIV cascade, especially among populations most likely to be left behind
%     by current HIV programs, may go undetected despite a country acheiving 95-95-95 overall.
% SM: would set up the idea here about how can have an overall cascade,
%     but with inequalities across populations - but perhaps that is what you were trying to do?
% HM: What does this mean? The lowest cascades among whom?
% LW: Unclear what you mean by lowest cascade?
% JK: Hopefully the added text before \ie clarifies?
\par
Numerous transmission modelling studies have sought to estimate
the prevention impacts of achieving 90-90-90+ across Sub-Saharan Africa \cite{Knight2022sr}.
Modelled populations are often stratified by risk,
% HM: Risk of disease acquisition?
% JK: acquisition and/or transmission -- I think it is sufficiently implied?
including key populations like female sex workers (FSW) and their clients,
% HM: This is the first time FSW appears
% JK: added spelled out
to capture important epidemic dynamics related to risk heterogeneity \cite{Watts2010}.
However, these studies have generally assumed that ART cascade
attainment (\ie proportions diagnosed, treated, and virally suppressed) or
progression (\ie rates of diagnosis, treatment initiation, and viral suppression)
were equal across modelled subpopulations.
For example, among the studies in \cite{Knight2022sr},
key populations were usually assumed to have
equal cascade progression with the population overall,
or greater in some scenarios, but never lesser.
% LW: Is this applicable to all 3 cascades? Rate of dx, tx initiation and tx discontinuation?
%     But never 'below average' made it sound like a limitation;
%     even though it may not be in the context of rate of discontinuation.
%     I would rephrase this sentence as sth below to be more specific:
%    "key populations were usually assumed to have "average" cascade progression as compared to the population overall;
%     a few studies had considered greater rates of diagnosis in xx,  greater treatment initiation in xx,
%     and/or greater treatment discontinuation in xx."
%     Cite literature in each case; so it is more explicit. For example. For treatment discontinuation,
%     I would not blame if nobody had assumed below average discontinuation rate for key population.
% JK: The discontinuation part is tricky since it does invert the association with VS attainment as you say,
%     however I think it could be confusing / distracting  to dive into these details here
%     (n.b. we don't mention 'discontinuation' yet here) vs focusing on 'viral suppression' in general
%     Some models also don't consider any 'off-ART' state, so % VS really is defined by the one rate.
Thus, the potential influence of intersecting risk and cascade inequalities
on ART prevention impacts has not been explored.
\par
We therefore examined the following questions in an illustrative modelling analysis:
\begin{enumerate}
  \item\label{obj:art.1} How are estimates of population-level ART prevention impacts
    % RK: impact or impacts? Style thing, no right answer
    % JK: I've searched to ensure 'impacts'
    influenced by inequalities in ART cascade across subpopulations?
    % SB: Is there a different way of calling folks other than risk groups.
    %     The answer may be no, but I know there may be some tension with this language
    %     given the common stigma associated with the term "risk" and being "high risk",
    % LW: Maybe population subgroups? And somewhere we define population subgroups as subgroups at different risk of HIV...
    % JK: Have tried to replace with subpopulations throughout --
    %     the downside is we lose the connection to 'risk heterogeneity'
    %     TODO: appendix too
  \item\label{obj:art.2} Under which epidemic conditions
    do such inequalities have the largest influence?
\end{enumerate}
We examined these questions using
a deterministic compartmental model of heterosexual HIV transmission in Eswatini,
focusing on inequalities related to sex work.
Eswatini has the highest national HIV prevalence in the world,
% SM: given our positionality here, I think ok to remove "but" and "outstanding"
% JK: good points!
and recently surpassed \cashi with minimal inequalities across subpopulations
\cite{SHIMS3,UNAIDS2023}.
As such, we used observed ART cascade scale-up in Eswatini as a \emph{base case}
reflecting exemplary and evidently attainable scale-up,
and explored \emph{counterfactual} scenarios in which scale-up was slower,
and where specific subpopulations could have been left behind.
% SM: re-read a few times, and I wonder if we should remove quotation marks around left behind;
%     quoations may imply not-real/fake/a-term-we-made-up?
% JK: sounds good -- I was waffling between these myself while writing...
% SM: something to consider for througout our paper,
%     to use term populations instead of risk groups (except in methods).
%     the term risk group is starting to be removed from HIV discussions, and Stef made a nice suggestion about this
%     in one of our grants and it read nicely to talk about populations
%     (esp b/c at times, FSW are not at highest per-capita risk of HIV acquistion due to programs! )
% JK: as noted above, have replaced with subpopulations throughout
