\section{Introduction}\label{intro}
% \begin{itemize}
%   \item global burden of HIV, role of ART in prevention
%   \item scale-up of ART, UTT, mixed trial results
%   \item risk heterogeneity theory, intersections with ART barriers, key pops, evidence of gaps
% \end{itemize}
Despite sustained efforts and advances in HIV prevention,
2021 saw an estimated 1.5 million new HIV infections globally \cite{AIDSinfo}.
A core pillar of HIV prevention remains treatment via antiretroviral therapy (ART).
Since 2016, WHO has recommended universal ART eligibility~\cite{WHO2016ART},
and efforts to scale-up ART coverage around the world continue \cite{AIDSinfo}.
Such efforts are often measured through the ART cascade,
as reflected in the UNAIDS \mbox{90-90-90} \cite{909090} and \mbox{95-95-95} \cite{959595} targets,
corresponding to the percentage of people living with HIV who know their status,
of whom, the percentage who are on ART,
of whom, the percentage who have undetectable viral load.
\par
Numerous modelling studies have sought to estimate
the prevention impacts of ART scale-up in Sub-Saharan Africa,
usually quantified as incidence reduction or cumulative infections averted
in scenarios with higher cascade attainment versus scenarios with lower attainment
\cite{Delva2012,Eaton2014a,Knight2022-sr}.
Such studies often stratify modelled populations by risk,
including key populations like female sex workers and their clients,
to capture important epidemic dynamics related to risk heterogeneity
\cite{Knight2022-sr,Garnett1996}.
These studies almost always assume that cascade attainment, or rates of cascade progression
--- rates of diagnosis, treatment initiation, and treatment failure/discontinuation ---
are equal across modelled risk groups \cite{Knight2022-sr}.
\par
Yet, there is growing evidence of differential ART cascade across population strata
\cite{Hakim2018,Green2020}.
These differences can be driven by
unique barriers to engagement in care faced by vulnerable populations,
which intersect with drivers of HIV risk \cite{Wanyenze2016,Schwartz2017,Schmidt-Sane2022}.
These intersections of risk and cascade heterogeneity
could potentially undercut the prevention impacts of ART scale-up
anticipated from model-based evidence \cite{Baral2019}.
Therefore, we sought to examine the following questions
in an illustrative modelling analysis:
\begin{enumerate}
  \item\label{obj:1} How are projections of ART prevention impacts influenced by
    differences in ART cascade across risk groups?
  \item\label{obj:2} Under which epidemic conditions do such differences have the largest influence?
\end{enumerate}
We examined these questions using
a deterministic compartmental model of heterosexual HIV transmission in eSwatini \cite{Knight2019},
focusing on differential risk related to sex work.
eSwatini has the highest national HIV prevalence in the world \cite{UNAIDS2021},
but has recently achieved outstanding cascade gains --- surpassing 95-95-95 ---
through multiple interventions led by the MaxART Consortium \cite{Walsh2020,AIDSinfo}.
As such, we use observed ART scale-up in eSwatini as a gold-standard base case for what is possible,
and explore counterfactual scenarios in which scale-up is slower,
and specific risk groups may be left behind.
