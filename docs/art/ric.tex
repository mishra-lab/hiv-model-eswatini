\newcommand{\pu}[2]{#1\,+\,#2}
\begin{ric}
  % SM: lets chat brieflly Thurs or Mon about the cover letter and RIC
  %     re: my suggestion to strengthen framing as almost like an impact evalaution
  %     (essentialy - you asked: what was the impact of achieving equality?)
  %     I think that, especially for audeince/journals like Lancet HIV/AIDS, etc.
  %     this framing works much better b/c of the data-informed nature of the modeling.
  %     Framing more like a fundamental modeling piece (like the turnover paper - a great piece!)
  %     will have a harder time getting past the editorial desk at these journals is my sense
  %     (remember like our serosorting modeling paper with Lancet HIV?),
  %     vs using this type of framing for foundational journals
  %     like PNAS, IJE/AJE, Epidemics, Nature Comms, etc.
  \paragraph{Evidence before this study}
  We adapted and updated a scoping review of HIV transmission modelling studies examining
  the prevention impacts of antiretroviral therapy (ART) cascade scale-up in Sub-Saharan Africa
  (full details in Appendix~\ref{sr.ric}).
  Our search yielded (\pu{prior}{update}) \pu{1367}{373} unique studies, of which:
  \pu{7}{5} examined the prevention impacts of achieving
  the UNAIDS 90-90-90 goals or greater (Table~\ref{tab:ric.refs}).
  Some of these studies considered differences in baseline rates of
  diagnosis, treatment initiation, and/or treatment failure/discontinuation
  --- mainly by sex, age, and occasionally risk.
  Three studies specifically examined scenarios of unequal cascade attainment across subpopulations,
  % JK: reverted to "scenarios of unequal cascade" from "scenarios with inequalities in cascade"
  %     b/c although it is less consistent with terminology elsewhere,
  %     I think it is slightly more precise?
  predicting substantially more infections when subpopulations at higher risk were left behind.
    However, these studies did not maintain consistent overall cascade across scenarios
  --- \ie each scenario reached a different cascade for the population overall ---
  % SM: not clear what we mean by "consistent population overall attainment"
  %     rephrase to make it easier for reader (its ok to use more words here if we need)
  and considered only hypothetical scenarios.
  \paragraph{Added value of this study}
  We developed and calibrated a detailed model of heterosexual HIV transmission
  % SM: check verb tense (I think usually past-tense used in RICs re: "what we did"/"what we found")?
  % JK: sounds good -- I've seen a bit of both, but happy to stick with past tense
  and observed ART cascade scale-up in Eswatini,
  drawing on population-level and female sex worker (FSW)-specific surveys.
  Eswatini achieved 95-95-95 overall and similar among FSW by 2020.
  % SM: even in RIC, explain why studying Eswatini.
  We estimated what would have happened if
  overall cascade scale-up had been slower
  (reaching only \casmd by 2020), and
  inequalities in the cascade among FSW and/or clients had not been addressed
  (reaching \caslo by 2020).
  We found that such cascade inequalities
  % SM: avoid "show" and use "found/identified", etc.
  %     But didn't all our counterfactuals have weaker cascades?
  %     (we do not really mention this in the abstract or cover letter re: "even for overall high")
  %     I suggest use numbers here (X-X-X), etc.
  %     otherwise, esp for non-modeler reader, I think this reads too vague for them.
  % JK: propose to remove above, as it's a bit redundant with the previous sentence?
  would have led to 50--167\% additional HIV infections in the overall population by 2030,
  as compared to equitable scale-up.
  % SM: speak in "would have" terms when estimating past impact
  \paragraph{Implications of all the available evidence}
  % SM: this should more closely match what we say in revised discussion.
  The population-level prevention benefits of achieving 95-95-95 overall
  % SM: here I think we can use present-tense.
  are undermined by inequalities in the HIV cascade, especially when
  populations experiencing disproportionate transmission risks are left behind,
  including but not limited to FSW and their clients.
\end{ric}
