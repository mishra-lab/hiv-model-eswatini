\newcommand{\pu}[2]{#1\,+\,#2}
\begin{ric}[b]
  \paragraph{Evidence before this study}
  We adapted and updated a scoping review of HIV transmission modelling studies examining
  the prevention impacts of antiretroviral therapy (ART) cascade scale-up in Sub-Saharan Africa.
  Our search yielded (\pu{prior}{update}) \pu{1367}{373} unique studies, of which:
  \pu{7}{5} examined the prevention impacts of achieving
  the UNAIDS 90-90-90 goals or greater (Table~\ref{tab:ric.refs}).
  Some of these studies considered differences in baseline rates of
  diagnosis, treatment initiation, and/or treatment failure/discontinuation
  --- mainly by sex, age, and occasionally risk.
  Three studies specifically examined scenarios of unequal cascade attainment across subpopulations,
  predicting substantially more infections when subpopulations at higher risk were left behind.
    However, these studies did not maintain consistent overall cascade across scenarios
  --- \ie each scenario reached a different cascade for the population overall ---
  and considered only hypothetical scenarios.
  \paragraph{Added value of this study}
  We developed and calibrated a detailed model of heterosexual HIV transmission
  and observed ART cascade scale-up in Eswatini,
  drawing on population-level and female sex worker (FSW)-specific surveys.
  Eswatini achieved 95-95-95 overall and similar among FSW by 2020.
  We estimated what would have happened if
  overall cascade scale-up had been slower
  (reaching only \casmd by 2020), and
  inequalities in the cascade among FSW and/or clients had not been addressed
  (reaching \caslo by 2020).
  We found that such cascade inequalities
  would have led to 40--149\% additional HIV infections in the overall population by 2030, % MAN
  as compared to equitable scale-up.
  \paragraph{Implications of all the available evidence}
  The population-level prevention benefits of achieving 95-95-95 overall
  are undermined by inequalities in the HIV cascade, especially when
  populations experiencing disproportionate transmission risks are left behind,
  including but not limited to FSW and their clients.
\end{ric}
