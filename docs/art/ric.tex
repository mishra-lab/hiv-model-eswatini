\newcommand{\pu}[2]{#1\,+\,#2}
\begin{ric} %SM: lets chat brieflly Thurs or Mon about the cover letter and RIC re: my suggestion to strengthen framing as almost like an impact evalaution (essentialy - you asked: what was the impact of achieving equality?). I think that, especially for audeince/journals like Lancet HIV/AIDS, etc. - this framing works much better b/c of the data-informed nature of the modeling. Framing more like a fundamental modeling piece (like the turnover paper - a great piece!) - will have a harder time getting past the editorial desk at these journals is my sense (remember like our serosorting modeling paper with Lancet HIV?), vs using this type of framing for foundational journals like PNAS, IJE/AJE, Epidemics, Nature Comms, etc. 
  \paragraph{Evidence before this study}
  We adapted and updated a scoping review of HIV transmission modelling studies examining
  the prevention impacts of antiretroviral therapy (ART) cascade scale-up in Sub-Saharan Africa
  (full details in Appendix~\ref{sr.ric}).
  Our search yielded (\pu{prior}{update}) \pu{1367}{373} unique studies, of which:
  \pu{7}{5} examined the prevention impacts of achieving
  the UNAIDS 90-90-90 goals or greater (Table~\ref{tab:ric.refs}).
  Some of these studies considered differences in baseline rates of
  diagnosis, treatment initiation, and/or treatment failure/discontinuation
  --- mainly by sex, age, and occasionally risk.
  Three studies specifically examined scenarios with inequalities in cascade attainment across subpopulations,
  predicting substantially more infections when subpopulations at higher risk were left behind.
  However, these studies did not maintain consistent population overall attainment across scenarios, %SM: not clear what we mean by "consistent population overall attainment" - rephrase to make it easier for reader (its ok to use more words here if we need)
  and considered only hypothetical scenarios. 
  \paragraph{Added value of this study}
  We developed and calibrated a detailed model of heterosexual HIV transmission %SM: check verb tense (I think usually past-tense used in RICs re: "what we did"/"what we found")?
  and observed ART cascade scale-up in Eswatini,
  drawing on population-level and female sex worker (FSW)-specific surveys.
Eswatini has achieved 96-95-95 overall and among FSW. We estimated what would have happened
if there had been a weaker overall cascade with inequalities in the cascade among FSW and/or clients of FSW. %SM: even in RIC, explain why studyiing Eswatini. 
 We found that even for high overall cascade attainment, %SM: avoid "show" and use "found/identified", etc.. But didn't all our counterfactuals have weaker cascades? (we do not really mention this in the abstract or cover letter re: "even for overall high") - I suggest use numbers here (X-X-X), etc. otherwise, esp for non-modeler reader, I think this reads too vague for them.
  cascade inequalities across subpopulations experiencing differential HIV risk
  --- namely lower cascade among FSW and their clients ---
  would have led to X-X percent additional HIV infections in the overall population between X-X[year]. %SM: speak in "would have" terms when estimating past impact
  \paragraph{Implications of all the available evidence} %SM: this should more closely match what we say in revised discussion.
The population-level prevention benefits of acheiving 95-95-95 overall is undermined  %SM: here I think we can use present-tense.
by inequalities in the HIV cascade, especially when 
populations experiencing disproportioante transmission risks are left behind, 
including but not limited to FSW and their clients.
\end{ric}
