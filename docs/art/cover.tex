\newgeometry{margin=30mm}% TEMP
\address{
  Dr. Michael S Saag\\
  Editor-in-Chief\\
  AIDS
}{Dr. Sharmistha Mishra\\
  MAP Centre for Urban Health Solutions\\
  Unity Health Toronto, University of Toronto}
Dear Editors,
\par
We are pleased to submit our manuscript entitled
\emph{Evaluating the impact of achieving cascade equality in Eswatini:
  a modelling study on the prevention impacts of antiretroviral therapy}
for consideration to publish as an \emph{Original Paper} in \emph{AIDS}.
\par
Countries continue to achieve and surpass
the UNAIDS 95-95-95 antiretroviral therapy (ART) cascade targets.
Yet, there is evidence of cascade inequalities in many contexts,
especially for subpopulations at greater risk for HIV acquisition and transmission,
such as female sex workers (FSW) and their clients.
A key question thus remains to what extent
such cascade inequalities could undermine the anticipated impact of achieving 95-95-95.
We systematically reviewed the existing literature
and found minimal evidence to answer this question.
\par
In this paper, we examine the potential impact of cascade inequalities
through a retrospective impact evaluation of ART scale-up in Eswatini.
We develop a model using population-level and FSW-specific surveys,
and in partnership with community leaders and program implementers.
These data suggest that Eswatini achieved 95-95-95 by 2020
not only among the population overall, but also among FSW specifically.
Our base case scenario thus reflects achieved cascade scale-up,
with minimal inequalities across subpopulations,
alongside other observed conditions (\eg increasing condom use).
We then compare this base case to four counterfactual scenarios in which
overall cascade was weaker (reaching only \casmd by 2020) and where
FSW, clients, both, or neither were disproportionately left behind
(reaching only \caslo by 2020).
Finally, we conduct sensitivity analyses to identify epidemic conditions
which influence the impact of cascade inequalities.
\par
% MAN below
We estimate that a weaker but equal cascade
would have led to 6.3--10.9\% more HIV infections in Eswatini by 2020;
whereas a weaker cascade that disproportionately left behind FSW and their clients
would have led to 10.8--18.6\% more infections
--- a 31--128\% increase.
That is, addressing cascade inequalities in Eswatini
through tailored programs for FSW and other subpopulations
has helped avert a substantial proportion of infections.
We also estimate that the impact of
leaving behind FSW and/or clients was largely determined by
their population sizes and HIV incidence ratio among clients \vs men overall.
\par
To our knowledge, this is the first modelling study to estimate
the potential impact of cascade equality versus inequalities across subpopulations
with consistent population-overall attainment across scenarios ---
\ie all counterfactual scenarios reach the same population-overall cascade by 2020.
The findings offer unique, data- and community-informed modelling insights,
using a real-world context where FSW cascade equality was achieved,
about the importance of equitable scale-up
for the maximizing the prevention impacts of ART.
\par
Thank you for your consideration and we look forward to hearing from you.
\medskip\par
Sincerely,
\par
Jesse Knight, PhD \& Sharmistha Mishra, MD, PhD\\
on behalf of all authors
\restoregeometry% TEMP
