\address{
  Peter Hayward\\
  Editor-in-Chief\\
  The Lancet HIV
}{Dr. Sharmistha Mishra\\
  MAP Centre for Urban Health Solutions\\
  St. Michael's Hospital, Unity Health Toronto\\
  University of Toronto}
Dear Peter Hayward,
% JK: @SM, I tried to incorporate 'inequalities' framing here too,
%     but in some places it was too awkward.
%     If you want/can make it work, feel free, but I also think it is OK as-is.
\par
We are pleased to submit our manuscript entitled
\emph{Intersections of risk and intervention heterogeneity:
  a modelling study on the prevention impacts of antiretroviral therapy in Eswatini}
for consideration as an \emph{Article} in \emph{The Lancet HIV}.
\par %SM: nice!
Five countries have now achieved
the UNAIDS 95-95-95 antiretroviral therapy (ART) cascade targets:
Botswana, Eswatini, Rwanda, Tanzania, and Zimbabwe.
Achieving these targets was projected by
mathematical models to reduce HIV incidence towards local elimination.
However, alongside efforts to rapidly increase coverage,
evidence is mounting of inequalities in cascade attainment,
including and especially for subpopulations
at higher and disproportionate risk for HIV acquisition and transmission, such as female sex workers (FSW) and their clients. 
The potential implications of these inequalities on
the prevention impacts of ART scale-up have not been examined.
\par
In this paper, we systematically examine the potential impact of inequalities on HIV transmission in the overall population,
focusing on inequalities among FSW and their clients in Eswatini. 
Drawing on population-level and FSW-specific surveys,
we build and calibrate a large compartmental transmission model
to reflect the observed HIV epidemic and ART cascade scale-up in Eswatini.
Because Eswatini also achieved 95-95-95 among FSW alongside acheiving 
95-95-95 overall, the setting provided an opportunity to more rigorously examine what if 
there had been inequalities (as is the case in many other settings). 
Using well-calibrated models to examine what could have happened in the past 
allows for other factors (such as scale-up of condom use, etc.) to also impart their effect on outcomes
when examining inequalities in the cascade (rather than make additional assumptions about future trajectories of interventions). %SM: something like this to bring out the strengths of approach 
Specifically, we examined what would have happened to overall HIV infections over time, 
if there had been a weak cascade overall (e.g. 80-8X-8X, as in other settings) with and without inequalities. That is,
what if FSW and/or clients had been left behind in the HIV cascade of care in Eswatini? 
We then conducted sensitivity analyses to explore and identify epidemic conditions which modify
the effect of cascade inequalities among FSW and clients on additional infections overall.
\par
We found that unequal ART cascade scale-up which left behind FSW or their clients %SM: suggest past tense
would have resulted in substantially more HIV infections \vs equitable scale-up.
In Eswatini, this means that an additional X-X percent additional infections if, %SM: suggest giving range from our results to make this matter more :) 
instead of the successful acehievement of 95-95-95 overall and in FSW
with tailored programs for FSW, there had been a weaker cascade with inequalities among FSW/clients.
That is, the past impact of addressing inequalities could have been as large as X-X percent, dependign on who was left behind.
Importantly, the impact on additional infections of
leaving behind FSW and/or clients was largely determined by
characteristics of the client population.
\par
To our knowledge, this is the first modelling study to explore
the potential implications of cascade inequalities across subpopulations
within consistent population-overall attainment across scenarios. %SM: could not follow this last part of the sentence, edit for clarity?
The findings offer unique, data-informed modelling insights, using a real-world context where FSW cascade equaliity was achieved,
about the importance of equitable scale-up
for the maximizing the prevention impacts of ART.
\par
Thank you for your consideration and we look forward to hearing from you.
\medskip\par
Sincerely,
\par
Jesse Knight and Sharmistha Mishra\\
on behalf of all authors
