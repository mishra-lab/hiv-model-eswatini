\section{Methods}\label{meth}
We constructed a deterministic compartmental model of heterosexual HIV transmission,
stratified by sex, sexual activity, HIV natural history, and ART cascade of care.
The model includes 8 risk groups,
including higher and lower risk female sex workers (FSW), and higher and lower risk clients of FSW,
and 4 partnership types, including regular and occasional sex work (Figure~\ref{fig:model}).
We calibrated the model to reflect the HIV epidemic and ART scale-up in eSwatini (base case).
We then explored counterfactual scenarios in which
ART cascade was reduced among various combinations of risk groups,
and quantified ART prevention impacts by comparing base case and counterfactual scenarios.
%===================================================================================================
\subsection{Model Parameterization \& Calibration}\label{meth.param}
Complete model details are given in Appendix~\ref{a:model},
and all code is available online.%
\footnote{\hreftt{github.com/mishra-lab/hiv-fsw-art}}
\paragraph{HIV}
Our model of HIV natural history included acute infection and stages defined by CD4-count.
We modelled relative rates of infectiousness by stage
as an approximation of viral load \cite{Wawer2005,Boily2009,Donnell2010},
as well as rates of HIV-attributable mortality \cite{Badri2006,Anglaret2012,Mangal2017}.
\paragraph{Risk heterogeneity}
We captured risk heterogeneity through risk group-level factors, including
group size, average duration in group, STI symptom prevalence,
and numbers / types of partnerships per year;
and partnership-level factors, including
assortative mixing, partnership duration, frequency of vaginal / anal sex, and levels of condom use.
Table~\ref{tab.het} summarizes key parameter values and sampling distributions
related to risk heterogeneity.
Only condom use varied (increased) over time (Figure~\ref{fig:fit.condom}).
To parameterize higher versus lower risk FSW, we conducted exploratory analysis of
survey data from Swazi FSW in 2011 \cite{Baral2014} and 2014 \cite{EswKP2014} (Appendix~\ref{a:model.fsw}).
We parameterized the remaining risk groups using reported data from national studies in
2006 \cite{SDHS2006}, 2011 \cite{SHIMS1}, and 2016 \cite{SHIMS2}.
We modelled expansion of voluntary medical male circumcision \cite{SHIMS2},
but did not model other interventions (e.g. ongoing pre-exposure prophylaxis scale-up \cite{EswCOP21})
nor non-heterosexual HIV transmission.
\begin{table}
  \centering
  \caption{Model parameters related to risk heterogeneity}
  \footnotesize
% TODO: update this with revisions
\begin{tabular}{llRClc}
  \toprule
                          &                                &  \multicolumn{3}{c}{Prior}   &         \\
  \cmidrule(rl){3-5}
  Parameter               & Stratification                 & Mean & (95\% CI)  & Distrib. &  Ref. * \\
  \midrule
  Population size         & FSW of women overall           &  2.8 & (0.6,~6.5) & Beta     &         \\
  (\% of total)           & Clients of men overall         &   30 & (6.0,~70)  & ---      & See~XX  \\
                          & HR FSW of FSW overall          &   20 &    ---     & ---      & Assumed \\
                          & HR clients of clients overall  &   20 &    ---     & ---      & Assumed \\[1ex]
  Duration in group       & HR FSW                         &  3.6 & (1.9,~5.8) & Gamma    & See~XX  \\
  (mean years)            & LR FSW                         &   10 & (9.0,~11)  & Gamma    & See~XX  \\
                          & All clients                    &   10 & (6.0,~15)  & Gamma    &         \\[1ex]
  Relative infectiousness & Acute infection                &  5.3 & (1.0,~13)  & Gamma    &         \\
                          & Any GUD p12m                   &  2.9 & (1.0,~5.7) & Gamma    &         \\[1ex]
  Relative susceptibility & Receptive vaginal sex          & 1.45 & (1.0,~2.0) & Gamma    &         \\
                          & Receptive anal sex             &   10 &    ---     & ---      &         \\
                          & Any GUD p12m: women            &  5.3 & (1.5,~12)  & Gamma    &         \\
                          & Any GUD p12m: men              &  7.7 & (2.0,~18)  & Gamma    &         \\[1ex]
  Any GUD p12m            & LR FSW                         &   16 &  (7,~28)   & Beta     &         \\
  prevalence (\%)         & HR FSW                         &   47 &  (19,~89)  & ---      &         \\
                          & HR clients                     &   12 &  (7,~22)   & ---      &         \\
                          & Everybody else                 &    7 &    ---     & ---      &         \\[1ex]
  Sex acts per            & Main/spousal                   &   78 & (27,~156)  & Gamma    &         \\
  partnership-year        & Casual                         &   30 & (4.4,~82)  & ---      & See~XX  \\
                          & Occasional sex work            &   12 &    ---     & ---      & Assumed \\
                          & Regular sex work               &   31 &  (18,~48)  & Gamma    &         \\[1ex]
  Partnership anal sex    & Main/spousal \& casual         &  5.9 & (0.6,~17)  & Beta     &         \\
  (\% of acts)            & Occasional \& regular sex work &  9.7 & (0.6,~29)  & Beta     &         \\[1ex]
  Condom use in 2020      & Main/spousal                   &   42 &  (31,~54)  & Beta     & See~XX  \\
  (\% of acts protected)  & Casual                         &   69 &  (65,~74)  & Beta     & See~XX  \\
                          & Occasional sex work            &   88 &  (78,~97)  & Beta     & See~XX  \\
                          & Regular sex work               &   79 &  (64,~90)  & Beta     & See~XX  \\
                          & Anal vs vaginal sex            &   77 &  (50,~95)  & Beta     &         \\[1ex]
  Partnerships per year   & LR FSW, occasional sex work    &   49 &  (30,~72)  & Gamma    &         \\
                          & HR FSW, occasional sex work    &   98 & (58,~153)  & ---      &         \\
                          & LR FSW, regular sex work       &  101 & (73,~133)  & Gamma    &         \\
                          & HR FSW, regular sex work       &  151 & (107,~205) & ---      &         \\[1ex]
  Sex work visits         & LR clients                     &   26 &  (11,~50)  & Gamma    &         \\
  per year                & HR clients                     &   89 & (34,~174)  & Gamma    &         \\
  % TODO: mixing!
  \bottomrule
\end{tabular}
\floatfoot
Notation ---
FSW: female sex workers;
Clients: of FSW;
HR: higher risk;
LR: lower risk;
p12m: past 12 months.
  \label{tab.het}
\end{table}
\paragraph{ART cascade}
We modelled rates of HIV diagnosis among people living with HIV as monotonically increasing over time,
with a base rate defined for low/medium activity women,
and fixed relative rates for low/medium activity men ($<1$), clients ($<1$), and FSW ($>1$).
We modelled ART initiation similarly except:
the relative rate for ART initiation among FSW was $<1$, and
we defined an additional relative rate by CD4 count ($\le1$)
to reflect historical ART eligibility criteria. % TODO: cite
We modelled viral suppression using a fixed rate for all groups,
corresponding to an average of 4 months on ART. % TODO: cite
We modelled treatment failure / discontinuation with a single monotonically decreasing rate
applied to all risk groups in the base case,
reflecting improving treatment success / retention over time.
Individuals with treatment failure / discontinuation could re-initiate ART at a fixed rate,
reflecting re-engagement in care or detection of treatment failure and initiation of alternative regimens.
We modelled rapid CD4 recovery during the first 4 months of ART,
followed by slower recovery while virally suppressed \cite{Battegay2006,Lawn2006,Gabillard2013}.
We modelled reduced HIV-attributable mortality among individuals on ART,
in addition to mortality benefits of CD4 recovery.
\paragraph{Calibration}
We calibrated the model to reflect
available data from eSwatini on HIV prevalence, HIV incidence, and ART cascade of care,
stratified by risk groups (Appendix~\ref{a:model.cal}) \cite{SDHS2006,SHIMS1,SHIMS2,Baral2014,EswKP2014}.
To reflect uncertainty in model inputs,
we sampled 100,000 sets of parameter values from specified prior distributions % MAN \x{n.cal}
(e.g. Table~\ref{tab.het}).
Parameters were sampled using Latin hypercube sampling \cite{McKay1979}
to obtain good coverage of parameter space,
except parameters with relational constraints
(e.g. condom use in casual partnerships must be higher than in main/spousal partnerships),
which were sampled randomly.
We computed the likelihood of each parameter set by
comparing model projections to calibration targets,
each target having an uncertainty distribution (Appendix~\ref{a:model.cal}).
We then selected the top \x{top.pct.fit}\% (\x{n.fit}) of parameter sets by likelihood (``model fits'')
for all subsequent analyses.
%===================================================================================================
\subsection{Scenarios \& Analysis}\label{meth.ana}
%---------------------------------------------------------------------------------------------------
\subsubsection{Objective 1: Influence of differences in cascade between risk groups}\label{meth.obj.1}
For Objective~\ref{obj:1},
we used calibrated model fits as the \emph{base case} scenario, reflecting
observed ART cascade scale-up in eSwatini~\cite{SDHS2006,SHIMS1,SHIMS2},
following WHO guidelines and reaching approximately 95-95-95 overall by 2020 \cite{AIDSinfo}.
We also assumed that FSW specifically achieved \mbox{95-95-95} by 2020, reflecting
trends in available data \cite{Baral2014,EswKP2014},
although the true FSW cascade in 2020 may have been lower.
\par
Next, we defined 4 \emph{counterfactual} scenarios in which
the overall population reached 60-80-80 by 2020, % TODO: justify this choice?
via reduced cascade progression among specific risk groups:
FSW, clients, and/or the remaining population (``lower risk'') (Table~\ref{tab:scenarios}).
We reduced cascade progression by applying a relative scaling factor $R$ to group-specific rates of:
diagnosis ($R_d\in[0,1]$), treatment initiation ($R_t\in[0,1]$),
and treatment failure / discontinuation ($R_f\in[1,25]$).
When FSW and/or clients had reduced cascade, we calibrated their $R$s so that
they achieved approximately 40-60-80 by 2020. % TODO: and this
By contrast, we calibrated $R$s for the lower risk population so that
the population \emph{overall} achieved 60-80-80 in all counterfactual scenarios,
thus ensuring that a consistent proportion of the overall population experienced reduced cascade.
When cascade rates among FSW and/or clients were unchanged from the base case,
the cascade they achieved by 2020 could be lower than 95-95-95
due risk group turnover and higher incidence.
All cascades continued to increase beyond 2020 due to
fixed rates of diagnosis, treatment initiation, and treatment failure / discontinuation thereafter.
\begin{table}[b]
  \centering
  \caption{Modelling scenarios for Objective~\ref{obj:1} defined by 2020 calibration targets}
  \label{tab:scenarios}
  \begin{tabular}{lCCCccc}
  \toprule
  & \multicolumn{3}{c}{ART cascade in 2020\tn{a}} 
  & \multicolumn{3}{c}{Re-scaled cascade rates\tn{b}} \\
  \cmidrule(rl){2-4}\cmidrule(rl){5-7}
  Scenario                                   &   FSW    & Clients  & Overall  & FSW & Clients & LR  \\
  \midrule
  \emph{Base Case}                           & 95-95-95 &    ---   & 95-95-95 & --- &   ---   & --- \\
  \emph{Leave Behind: FSW}                   & 40-60-80 &    ---   & 60-80-80 & \By &   \Bn   & \By \\
  \emph{Leave Behind: Clients}               &    ---   & 40-60-80 & 60-80-80 & \Bn &   \By   & \By \\
  \emph{Leave Behind: FSW \& Clients}        & 40-60-80 & 40-60-80 & 60-80-80 & \By &   \By   & \By \\
  \emph{Leave Behind: Neither}               &    ---   &    ---   & 60-80-80 & \Bn &   \Bn   & \By \\
  \bottomrule
\end{tabular}\floatfoot
\tnt[a]{Cascade: \% diagnosed among HIV+; \% on ART among diagnosed; \% virally suppressed among on ART};
\tnt[b]{Rates of: diagnosis; ART initiation; treatment failure}.
% TODO: explain counterfactual scenario re-fitting details?
\end{table}
\par
We quantified ART prevention impacts via
cumulative additional infections (CAI) and additional incidence (AI)
in the counterfactual scenarios versus the base case.
Differences (CAI and AI) were defined relative to the base case,
and computed over multiple time horizons up to 2040.
For each scenario, we computed ART prevention impacts for each model fit,
and reported median (95\% CI) values across model fits, reflecting uncertainty.
%---------------------------------------------------------------------------------------------------
\subsubsection{Objective 2: Conditions under which cascade differences matter most}\label{meth.obj.2}
For Objective~\ref{obj:2}, we aimed to estimate the effects of
lower ART cascades among certain risk groups on relative CAI and AI,
and potential effect modification by epidemic conditions.
For this analysis, we explored a wider range of counterfactual scenarios by randomly sampling
$R_d, R_t \sim \mathrm{Beta}(\alpha=1.8,\beta=1.2)$, and
$R_f \sim \mathrm{Gamma}(\alpha=1.33,\beta=3)$ for each of:
FSW, clients, and the remaining lower risk population (9 total values).
Sampling distributions were chosen to obtain cascades in 2020 spanning
approximately 75-80-95 through 95-95-95.
We sampled $N_k = \x{n.sens.per}$ sets of $R$ for each model fit using Latin hypercube sampling.
\par
For each $R$ sample $k$ and model fit (\x{n.sens} total parameter sets), we computed
the relative CAI and AI versus the base case, as in Objective~\ref{obj:1},
and the proportion virally unsuppressed ($U$) among people living with HIV by 2020,
as a summary measure of ART cascade gaps.
Then, we defined
$dU_k = U_k - U_0$ as the difference in virally unsuppressed for the population overall
in sample $k \in [1,10]$ versus the base case $k = 0$, and
$RU_{ki} = U_{ki} / U_{k,all}$ as the ratio of virally unsuppressed
among risk group $i$ versus the population overall in sample $k$.
\par
Next, we defined the following key measures of epidemic conditions ($C$) related to sex work:
FSW and client population sizes (\% population overall);
average duration selling sex (FSW) and buying sex (clients); and
HIV prevalence ratios in 2000 among FSW vs lower risk women, and among clients vs lower risk men.
For these measures, we combined higher and lower risk FSW, and likewise higher and lower risk clients.
We used HIV prevalence ratios to reflect
summary measures of risk heterogeneity under pre-ART conditions.
By contrast, including all modelled risk factors
for HIV acquisition and transmission (Table~\ref{tab.het})
in the analysis could lead to overfitting and improper inference due to effect mediation.
Finally, we fit a general linear model of the form:
\begin{equation}\label{eq:obj.2}
  CAI = \beta_0
      + \beta_d dU
      + \sum_i \beta_i~{RU}_i
      + \sum_i \sum_j \beta_{ij}~{RU}_i~{C}_j
\end{equation}
and determined the significance of relative unsuppression effects ($\beta_i$)
and effect modification by epidemic conditions ($\beta_{ij}$).
We used a generalized estimating equation to control for repeated use of each model fit.
We used standardized variables (${dU}$, $RU$, $C$)
$\hat{x}_k = (x_k - \mu_{x_k}) / \sigma_{x_k}$
to assess the relative influence of each variable.
