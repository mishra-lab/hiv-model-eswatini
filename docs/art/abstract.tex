\paragraph{Background}
Inequalities in the antiretroviral therapy (ART) cascade across subpopulations
remain an ongoing challenge in the global HIV response.
Eswatini achieved the UNAIDS 95-95-95 ART cascade targets by 2020,
with differentiated programs to minimize inequalities across subpopulations,
including for female sex workers (FSW) and their clients.
We sought to estimate the impacts of this achievement,
through a retrospective impact evaluation of ART scale-up in Eswatini.
\paragraph{Methods}
Drawing on population-level and FSW-specific surveys,
we developed a compartmental model of heterosexual HIV transmission,
and calibrated it to observed
HIV prevalence, incidence, and ART cascade scale-up in Eswatini.
We then defined four counterfactual scenarios in which
the population overall reached only \casmd by 2020,
but where FSW, clients, both, or neither
were disproportionately left behind, reaching only \caslo.
We estimated additional HIV infections by 2020
in counterfactual \vs observed scenarios,
and identified epidemic conditions which maximized differences.
\paragraph{Results}
Compared with observed cascade scale-up in Eswatini,
leaving behind neither FSW nor their clients led to median (95\% CI)
8.8~(6.3,~10.9)\% additional infections by 2020 \vs % MAN
14.3~(10.8,~18.6)\% if both were left behind % MAN
--- a 63~(31,~128)\% increase. % MAN
The impact of leaving behind FSW and/or clients was largely determined by
their population sizes and HIV incidence ratio among clients \vs men overall.
\paragraph{Conclusions}
Inequalities in the ART cascade across subpopulations
can undermine the anticipated prevention impacts of cascade scale-up.
As Eswatini has shown,
addressing inequalities in the ART cascade that intersect with transmission risk
can maximize incidence reductions from cascade scale-up.
