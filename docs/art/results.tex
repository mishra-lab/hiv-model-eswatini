\section{Results}\label{art.res}
% LW: "In the base case (calibrated to reflect the HIV epidemic features and HIV treatment cascade in Eswatini between year xx and xx),"
% SM: ok to say clients instead of clients of FSW (since we only have one definition of clients);
%     same comments as before re: populations vs groups;
Early epidemic emergence in Eswatini was driven by unmet needs in the context of regular sex work partnerships
% SM: lets do a full review re: language that may lead to inadvertent "people/partnerships" to blame,
%     and revise re: unmet needs in the context of... (I know more words but it has been helpful feedback for previous papers!).
%     also, how about either defining driver, or rewording to say that depended on...;
%     I'm also not sure the term "early epidemic emergence" will be clear to non-modelers?
%     emergence in Eswatini is due to unmet needs in context of sex work.
%     (but what if it was also related to what was happening in South Africa).
%     Am thinking we couch this slightly more cautiously... see what you think of these edits?
%     "In the modeled epidemic, transmission in the context of regular sex work partnerships was the most important driver of epidemic growth."
%     (that way we do not get into the issues of emergence, without connections to other countries,
%     being a stickiing point for reviewers and distracting from the main hypotheses/questions in the paper?)
% LW: Need to clarify here: do u mean causal contribution the most since 1997(from 1997 onward) or by 1997;
%     The latter felt after 1997, sth else contributed the most. Although I guess u meant the former.
(Figures \ref{fig:wiw.base.part}~and~\ref{fig:wiw.base.alluvial}).
However, transmissions in the context of casual partnerships contributed the majority of cumulative/annual infections by 1997, % MAN
% SM: cumulative infections by 1997 or annual infections in 1997?
%     suggest terms like 'transmissions in context of xxx partnership'
%     (am thinking about how we do not want to say specific partnerships are to "blame",
%     but that unmet needs within those contexts?) I obviously spend a lot of time thinking about
%     how to frame this as have been on multiple receiving ends of how my phrasing over the years has evolved
%     based on community and Stef and others' feedback on language ...
%     and is still a work in progress for me! so just thinking aloud here too...
% SB: Do we know these to be non-transactional? I always found the MCP (mutliple concurrent partnership) framing difficult
%     as we folks often were not asked about the dynamics of these partnerships even if not formal sex work.
including 44\% (median) of new infections in 2020 in the base case. % MAN
% SM: instead of calling this base case within sentences, how about this section starts with the basecase..
% HM: The sentence jumped from 1997 to 2020? Also, sex work regular seems contributed most of infection in early years?
%     Casual does not seem to have 44$ based on B.14
By 2020, infections among clients had directly and indirectly led to the largest number/fraction
of cumulative/annual(?) infections in the total population (Figure~\ref{fig:wiw.base.from})
% SM: lets avoid "X population transmitted infections" (i know used by modelers all the time,
%     and even Leigh and others, but am trying to put myself in the shoes of clients here... :).
%     also, are we including indirect transmissions and if so, would be worth mentioning?
and women at lower risk had acquired the most infections (Figure~\ref{fig:wiw.base.to}).
% SM: from clients directly/indirectly or overall?
%     I think would be good to split up this sentence to ensure reader doesn't misinterpret.
Overall HIV prevalence in 2020 was median (95\% confidence interval):
24.4~(23.3,~25.4)\% (Figure~\ref{fig:fit.prev}), % MAN
and overall incidence was 8.0~(6.7,~9.4) per 1000 person-years (Figure~\ref{fig:fit.inc}).
The prevalence ratio between FSW and women overall was 1.75~(1.69,~1.86), % MAN
and between clients and men overall it was 2.14~(1.55,~2.99) % MAN
% SM: so overall includes non-clients yes?
(Figure~\ref{fig:fit.prev1v2}).
% LW: I wonder if this should proceed the contribution part above
Due to turnover and higher HIV incidence among FSW,
achieving similar rates of diagnosis among FSW versus other women (Figure~\ref{fig:fit.dx.rate})
required approximately twice the rate of testing. % MAN
Sex work contributed a growing proportion of infections
over 2020--2040: from 40\% to 52\% (Figure~\ref{fig:wiw.base.part}). % MAN
%===================================================================================================
\subsection{Objective~1: Influence of cascade differences between risk groups}\label{art.res.1}
% SM: one suggestin to consider (but see what you think), might it help to call the scenairos A, B, C, D.
%     Probably not (too many more things to remember, but would just give it a thought).
%     I found myself sometimes still going back and forth a bit to Table 2 when reading this part of the results.
%     (though I don't know that A, B, C, D would change that! ;)
Figure~\ref{fig:art.1.cascade} illustrates cascade attainment over time
in each of the four counterfactual scenarios (\casmd overall by 2020),
and the base case (\cashi overall by 2020).
Figure~\ref{fig:art.1.inc} illustrates overall HIV incidence in each scenario.
Figure~\ref{fig:art.1.rai} then illustrates
cumulative additional infections (CAI) and additional incidence rate (AIR)
in each counterfactual scenario \vs the base case.
Leaving behind both FSW and clients led to the largest CAI: median (95\%~CI)
% SB: In other places you used quotation marks for each time you wrote "leaving behind".
26.7~(19.7,~33.8)\,\% more than the base case by 2030. % MAN
By contrast, the scenario with an overall cascade of 80-80-90 where neither
% SM: this is in the methods, but I think worth repeating here at least once.
FSW and nor clients were left behind led to the fewest additional infections:
13.3~(9.2,~18.7)\,\% more than the base case by 2030 --- % MAN
a 50.9~(33.5,~62.5)\,\% reduction. % MAN
Leaving behind either FSW or clients resulted in
22.3~(16.5,~27.9)\,\% or 19.1~(12.9,~25.0)\,\% additional infections
\vs the base case, respectively. % MAN
Results were similar for AIR.
Relative differences were similar for additional incidence rate.
The consequence, with respect to whom acquired most of the additional infections
differed depending on which population/group was left behind.
% SM: maybe rephrase this sentence, or break it up into two (edited for this option)
(Figure~\ref{fig:art.1.wiw.to}).
% LW: Interesting. ould benefit some clarification –see if my edits reflects what you mean here.
%     And provide numbers would be helpful.
%     "In the scenario when FSW were left behind, the most additional infections in the population
%     were attributed to infections among clients (xx%); in the scenario whern clients ..."
For example, most of the additional infections occured among clients when FSW were left behind,
whereas most of the additional infections occured among women at lower risk when clients were left behind.
The majority of additional infections were transmitted
% SM: in all scenarios (might be worth saying that?)
via casual partnerships in all scenarios (Figure~\ref{fig:art.1.wiw.part}). % MAN
\begin{figure}[h]
  \centering\includegraphics[width=.7\linewidth]{art.1.rai}
  \caption{Relative additional infections under counterfactual scenarios \vs the base case}
  \label{fig:art.1.rai}
  \floatfoot{\ffart; \ffbox.}
\end{figure}
%===================================================================================================
\subsection{Objective~2: Conditions that maximize the influence of cascade differences}\label{art.res.2}
The fitted regression models \eqref{eq:art.2.glm} indicated that
population-overall viral unsuppression ($D$) and
% Reading that word hurts me, but perhaps "lack of suppression" is too clunky?
group-specific unsuppression among FSW and clients ($d_i$)
were each strongly and positively associated with 2030 CAI and AIR outcomes ($p < 10^{-5}$),
though the effect size was larger for clients.
These associations support the results of Objective~\ref{obj:art.1}.
% Can you be more specific/ The effect was larger for clients part? Unclear
Figure~\ref{fig:art.2} plots the estimated effects of
group-specific unsuppression $d_i$, and
effect modification by epidemic conditions $C_j$.
% MAN below
The effect of unsuppression among both FSW and clients on CAI and AIR increased with:
larger client population size,
faster client turnover,
larger HIV prevalence ratio among clients \vs other men, and
smaller HIV prevalence ratio among FSW \vs other women.
The effect of unsuppression among FSW further increased slightly with
duration in sex work, while
the effect of unsuppression among clients further increased slightly with
\emph{smaller} FSW population size.
\begin{figure}[h]
  \subcapoverlap\centering
  \vskip1ex
  \begin{subfigure}{.7\linewidth}
    \includegraphics[width=\linewidth]{art.2.cai}
    \caption{\raggedright}
    \label{fig:art.2.cai}
  \end{subfigure}
  \vskip1ex
  \begin{subfigure}{.7\linewidth}
    \includegraphics[width=\linewidth]{art.2.air}
    \caption{\raggedright}
    \label{fig:art.2.air}
  \end{subfigure}
  \caption{Estimated effects on relative additional infections
    of disproportionate viral unsuppression ($d$) among FSW and clients \vs population overall,
    plus effect modification by epidemic conditions}
  \label{fig:art.2}
  \floatfoot{
    \sfref{fig:art.2.cai} cumulative additional infections,
    \sfref{fig:art.2.air} additional incidence rate
    by 2030 \vs base case;
    FSW: female sex workers; Clients: of FSW;
    PR: prevalence ratio in 2005;
    $d_i$: difference in group-$i$-specific viral unsuppression
      \vs population overall within counterfactual scenario;
  \ffpbar[effect estimated via \eqref{eq:art.2.glm}].}
\end{figure}
\pagebreak % TEMP
