\section{Objectives}\label{sr.art}
%===================================================================================================
\subsection{Objective~1}\label{sr.art.1}
Figure~\ref{fig:art.1.cascade} (next page) illustrates cascade attainment over time across
base case and counterfactual scenarios (Table~\ref{tab:art.1.scen})
for FSW, clients, everyone else, and the population overall.
Transient declines in VLS among treated around 2010 correspond to expanding ART eligibility.
Figure~\ref{fig:art.1.inc} illustrates overall HIV incidence over time across scenarios, while
Table~\ref{tab:art.1.num} summarizes the results of Objective~\ref{obj:art.1} numerically.
As in \sref{sr.cal.wiw}, Figure~\ref{fig:art.1.wiw} illustrates
the distributions of additional infections over time \vs the base case across counterfactual scenarios,
stratified by partnership type, transmitting group, and acquiring group.
\delayfigure{
\begin{figure}
  \centerline{\includegraphics[scale=.8]{art.1.cascade}}
  \caption{Cascade attainment over time across scenarios}
  \label{fig:art.1.cascade}
  \floatfoot{\ffpopz; \ffcas; \ffart; \ffribbon.}
\end{figure}}
\begin{table}[b]
  \caption{Numeric summary of outcomes for Objective~\ref{obj:art.1}}
  \label{tab:art.1.num}
  \setlength\tabcolsep{4pt}
\begin{tabular}{lrcrcrcrc}
  \toprule
  & \multicolumn{4}{c}{cumulative HIV infections}
  & \multicolumn{4}{c}{HIV incidence rate} \\
  \cmidrule(rl){2-5}\cmidrule(rl){6-9}
  & \multicolumn{2}{c}{raw (1000s)}       & \multicolumn{2}{c}{relative (\%)\tn{a}}
  & \multicolumn{2}{c}{raw (per 1000 PY)} & \multicolumn{2}{c}{relative (\%)\tn{a}} \\
  \cmidrule(rl){2-3}\cmidrule(rl){4-5}\cmidrule(rl){6-7}\cmidrule(rl){8-9}
  Scenario                           & Mean & (95\%~CI)  & Mean &    (95\%~CI) & Mean &  (95\%~CI)  & Mean & (95\%~CI)  \\
  \midrule
  \emph{Base Case}                    & 349 & (332,~372) &  --- &      ---     &  6.6 &  (5.3,~7.6)  & --- &     ---    \\
  \emph{Leave Behind: FSW}            & 388 & (362,~406) & 10.9 &  (8.4,~13.3) & 15.7 & (13.3,~18.2) & 140 & (106,~179) \\
  \emph{Leave Behind: Clients}        & 392 & (370,~409) & 12.4 &  (9.8,~14.6) & 17.3 & (14.4,~19.9) & 163 & (120,~201) \\
  \emph{Leave Behind: FSW \& Clients} & 400 & (381,~419) & 14.3 & (10.8,~18.6) & 19.1 & (15.2,~22.3) & 193 & (152,~235) \\
  \emph{Leave Behind: Neither}        & 381 & (353,~398) &  8.8 &  (6.3,~10.9) & 13.3 & (10.8,~15.7) & 106 &  (68,~138) \\
  \bottomrule
\end{tabular}
\floatfoot{\ffart;
  \tnt[a]{relative increase \vs base case as defined in \eqref{eq:art.out}
    --- \ie $(\txn{scenario} - \txn{base}) / \txn{base}$};
  all outcomes for 2020.}

\end{table}
\begin{figure}[h]
  \centerline{\includegraphics[scale=.8]{art.1.inc}}
  \caption{Overall HIV incidence over time across scenarios}
  \label{fig:art.1.inc}
  \floatfoot{\ffart; \ffribbon.}
\end{figure}
\clearpage
\begin{figure}
  \subcapoverlap
  \foreach \var in {ptr,from,to}{
  \begin{subfigure}{\linewidth}
    \includegraphics[width=\linewidth]{art.1.wiw.\var}
    \caption{\raggedright}
    \label{fig:art.1.wiw.\var}
  \end{subfigure}}
  \caption{Additional infections in each counterfactual scenario
    \vs the base case, stratified by:
    \sfref{fig:art.1.wiw.ptr} partnership type,
    \sfref{fig:art.1.wiw.from} transmitting group, and
    \sfref{fig:art.1.wiw.to} acquiring group}
  \label{fig:art.1.wiw}
  \floatfoot{\ffart; \ffwiw.}
\end{figure}
\clearpage % TEMP
%===================================================================================================
\subsection{Objective~2}\label{sr.art.2}
%---------------------------------------------------------------------------------------------------
\paragraph{Sensitivity Analysis Methodology}
The conventional method for sensitivity analysis of HIV transmission models is to
examine partial rank correlation coefficients (PRCC) of model inputs (parameters) \vs model outputs,
after using Latin hypercube sampling to obtain good coverage of plausible input values
\cite{Blower1994,Marino2008,Johnson2016cc}.
Estimating PRCCs then fundamentally uses a regression-like procedure \cite{Kim2015}.
Overall, this approach aims to make minimal assumptions:
first about the shape of relationship between model inputs and outputs
--- and thus only considers linear relationships among rank-transformed variables; and
second about the causal relationships among variables
--- and thus considers no interactions or hypothesized causal pathway.
\par
We aimed to improve on this method in two ways for Objective~\ref{obj:art.2}.
First, we do not rank transform any variables (inputs or outputs)
in order to retain their original scale --- \eg preferring
``1000 more cumulative infections'' \vs
``a 10\% increase in the rank of cumulative infections among samples''.
This forces the choice of a link function for regression,
for which we chose simple linear,
and later verified by examining residuals (Figure~\ref{fig:art.2.r}, below).
Second, we conceptualized a specific causal pathway
(Figure~\ref{fig:art.2.dag}) to guide our analysis.
In this conceptualization, the main predictors of each outcome ($Y$) are
overall viral non-suppression ($D$), and
relative non-suppression among FSW and clients \vs the population overall ($d_i$).
Next, we conceptualized epidemic conditions ($C_j$) as
effect modifiers of main predictors (interactions).
However, if all main predictors $D = d_i = 0$
(no differences in viral suppression \vs base case),
both outcomes are zero by definition in \eqref{eq:art.out};
thus, epidemic conditions cannot have main effects
--- in contrast to conventional PRCC analysis.
\begin{figure}[b]
  \centering\includegraphics[scale=1]{art.2.dag}
  \caption{Directed acyclic graph (DAG) for inferring
    the epidemic conditions under which
    differential viral suppression across subpopulations matters most}
  \label{fig:art.2.dag}
  \floatfoot{
    $Y$: cumulative additional infections or additional incidence rate by 2020;
    $D$: difference in population-overall viral non-suppression
      in counterfactual \vs base case scenario;
    $d_i$: difference in subpopulation-$i$-specific viral non-suppression
      \vs population overall within counterfactual scenario;
    $C_j$: epidemic conditions (effect modifiers of $d_i$).}
\end{figure}
%---------------------------------------------------------------------------------------------------
\paragraph{Supplemental Results}
Figure~\ref{fig:art.2.cascade} illustrates the distributions of cascade attainment by 2020
for FSW, clients, everyone else, and the population overall,
in the 10,000 randomly generated counterfactual scenarios
for Objective~\ref{obj:art.2}.
Figure~\ref{fig:art.2.r} then illustrates the simulated \vs regression-estimated
cumulative additional infections and additional incidence rate,
which supports the use of linear regression
despite minor heteroskedasticity.
\begin{figure}[h]
  \centering
  \includegraphics[scale=.8]{art.2.cascade}
  \caption{Cascade attainment by 2020 across 10,000 randomly generated counterfactual scenarios}
  \label{fig:art.2.cascade}
  \floatfoot{\ffpopz; \ffcas; \ffbox.}
\end{figure}
\begin{figure}[h]
  \begin{subfigure}{.5\linewidth}
    \includegraphics[width=\linewidth]{art.2.cai.r}
    \caption{Cumulative additional infections}
    \label{fig:art.2.cai.r}
  \end{subfigure}%
  \begin{subfigure}{.5\linewidth}
    \includegraphics[width=\linewidth]{art.2.air.r}
    \caption{Additional incidence rate}
    \label{fig:art.2.air.r}
  \end{subfigure}
  \caption{Simulated \vs regression-estimated outcomes,
    and corresponding residuals for Objective~\ref{obj:art.2}.}
  \label{fig:art.2.r}
  \floatfoot{Outcomes computed for 2020 \vs base case;
    regression models defined in \eqref{eq:art.2.glm}.}
\end{figure}
