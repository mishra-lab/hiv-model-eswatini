\title
  [Impact of cascade equality in Eswatini]
  {Evaluating the impact of achieving cascade equality in Eswatini:\newline
  a modelling study on the prevention impacts of antiretroviral therapy}
\author[1,2,*]{Jesse Knight}
\author[1]{Huiting Ma}
\author[3]{Bheki Sithole}
\author[4]{Lungile Khumalo}
\author[1]{Linwei Wang}
\author[5]{Sheree Schwartz}
\author[3]{Laura Muzart}
\author[6]{Sindy Matse}
\author[6,+]{Zandile Mnisi}
\author[7]{Rupert Kaul}
\author[8]{Michael Escobar}
\author[5]{Stefan Baral}
\author[1,2,7,8,9]{Sharmistha Mishra}
\affil[1]{MAP Centre for Urban Health Solutions, Unity Health Toronto}
\affil[2]{Institute of Medical Science, University of Toronto}
\affil[3]{EpiC, FHI 360, Eswatini}
\affil[4]{Voice of Our Voices, Eswatini}
\affil[5]{Bloomberg School of Public Health, Johns Hopkins University}
\affil[6]{Ministry of Health, Eswatini}
\affil[7]{Division of Infectious Diseases, Department of Medicine, University of Toronto}
\affil[8]{Dalla Lana School of Public Health, University of Toronto}
\affil[9]{Institute for Clinical Evaluative Sciences, Toronto, Ontario}
\affil[+]{\emph{in memory of}}
\affil[*]{Corresponding author: Jesse Knight,
  \hreftt[mailto:]{jesse.x.knight@protonmail.com}\newline
  209 Victoria Street, Toronto, Canada, M5B~1T8}
\metafield*{funding}{
  The study was supported by
  the Natural Sciences and Engineering Research Council of Canada (CGS-D);
  the Ontario Ministry of Colleges and Universities (QEII-GSST);
  the Canadian Institutes of Health Research (FN-13455);
  the National Institute of Allergy and Infectious Diseases (R01AI170249).}
\metafield{conflicts}{None declared.}
\metafield*{acknowledgements}{We thank:
  Kristy Yiu, Samantha Lo (Unity Health Toronto) for research coordination support;
  Amrita Rao, Carly Comins (Johns Hopkins University),
  Alex Whitlock, Korryn Bodner (Unity Health Toronto), and
  Leigh Johnson (University of Cape Town)
    for helpful discussions and feedback on model design.}
\metafield{contributions}{
  Conceptualization: JK,LW,SM;
  formal analysis: JK,HM,LW;
  investigation: all authors;
  methodology: JK,HM,BS,LK,LW,ME,SB,SM;
  project administration: JK,HM,LW,RK,ME,SB,SM;
  software: JK;
  supervision: RK,ME,SB,SM;
  validation: JK,SM;
  visualization: JK;
  original draft: JK,SM;
  review \& editing: all authors.
  All authors except ZM have read and approved the final manuscript.}
\metafield[Data \& Code]{code}{
  We used only published aggregate data,
  except for individual-level data from two female sex worker surveys,
  which were accessed under approval from
  \blind{the Scientific and Ethics Committee of Eswatini Ministry of Health (MH/599B)}, and
  \blind{the Institutional Review Board of the Johns Hopkins Bloomberg School of Public Health (3508)}.
  All code and selected results are available at:
  \blind{\hreftt{github.com/mishra-lab/hiv-model-eswatini}}}
\metabreak
\metafield{preprint}{
  medRxiv: \blind{\hreftt{https://doi.org/10.1101/2024.02.16.24302584}}}
\metafield{journal}{AIDS}
\metafield{revision}{R1}
\date{2025 April 13}
\abstract{\input{abstract}}
\metafield{keywords}{
  HIV,
  mathematical model,
  antiretroviral therapy,
  sex work,
  healthcare disparities,
  Southern Africa}
\metafield[Word Count]{words}{Abstract: 250\\Body: 3500}
