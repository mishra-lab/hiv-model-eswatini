\section{Proposed Approach: Effective Partnerships Adjustment}\label{foi.prop}
Considering the limitations outlined above, and
the potential drawbacks of alternate modelling frameworks described in \sref{sr.foi.alt},
an improved approach to modelling HIV transmission via sexual partnerships
within the traditional compartmental framework would be useful.
In this section, we propose such an approach:
the \emph{Effective Partnerships Adjustment} (EPA).%
\footnote{A preliminary version of this approach was presented in \cite{Knight2022smdm}.}
That is, this approach overcomes
the main limitations of prior approaches described above,
without the need to change modelling frameworks.
% TODO: (?) mention by name pair-based and individual-based models here
%===================================================================================================
\subsection{Conceptual Development}\label{foi.prop.concept}
A core challenge of modelling HIV transmission via sexual partnerships is
to account for inert sex acts.
Any partnership where transmission has already occurred
is seroconcordant and thus ``transmission ineffective'',
and so should be removed from the force of infection.%
\footnote{An illustrative scenario to highlight this issue is given in \sref{sr.foi.toy}.}
Our fundamental insights is that:
in a compartmental (non-pair-based) model,
these partnerships can be tracked as proportions of individuals, namely:
all individuals who recently acquired infection \emph{and}
all individuals who recently transmitted infection.
Here, we use ``recent'' to mean ``before individuals change partners''.
If some individuals have multiple concurrent partners,
then these individuals should not be removed entirely,
but their numbers of ``effective partnerships'' should be reduced by 1.
If multiple types of partnerships are considered,
then only the partnership type involved in the transmission should be reduced.
This adjustment to ``effective partnerships'' can then be applied
until these individuals change partners
% TODO: (*) but part of partnership has already passed, so delta_new < delta_total
--- at a rate inversely related to the partnership duration: $\delta^{-1}$.
However, during this period, these individuals can/should still be modelled
to progress as usual through different stages of infection, treatment, etc.
\par
Using this conceptual basis,
we propose a new stratification of the modelled infected population, denoted $\p$.
The stratum $\p = 0$ corresponds to no recent transmission,
or all ``new'' (potentially serodiscordant) partnerships.
Other strata $\p > 0$ correspond to recent transmission via (to or from) partnership type $\p$.
Figure~\ref{fig:model.hiv.p} illustrates the new stratification
together with with an existing HIV infection stratification (Figure~\ref{fig:model.hiv}).
Following infection, all individuals enter a stratum $\p > 0$
corresponding to the partnership type $p$ by which they were infected.
Thus, the rate of entry to this stratum among susceptibles is defined by
the force of infection without aggregating across partnership types: $\lambda_{p}$.
Individuals may then transition from $\p > 0$ to $\p = 0$
upon forming a new partnership, at a rate $\delta_p^{-1}$.
Finally, individuals may re-enter any stratum $\p > 0$
if they transmit infection via partnership type $p$.
We denote the corresponding rate as $\lambda'_{p}$,
representing the per-person rate of \emph{transmission},
not \emph{acquisition} as in $\lambda_{p}$.
This rate $\lambda'_{p}$ is not defined or needed in prior approaches (\sref{foi.prior})
but we develop the necessary equations below in \sref{foi.prop.eq}.
The issue of multiple post-transmission partnerships is discussed in \sref{foi.prop.mp}.
\begin{figure}
  \centering\includegraphics[scale=1]{model.hiv.p}
  \caption{Modelled states and transitions related to HIV infection,
    and a new stratification $\p$ to track
    the proportions of individuals in partnerships where transmission already occurred}
  \label{fig:model.hiv.p}
  \floatfoot{
    $S$: susceptible;
    $I_{h}$: infectious in stage $h$;
    $p$: partnership type;
    $\p$: new stratification, where
      $\p = 0$ reflects no recent transmission (all new partnerships), and
      $\p > 0$ reflects recent transmission via a type-$p$ partnership;
    $\lambda$: force of infection per susceptible;
    $\lambda'$: force of infection per infectious;
    $\eta$: rate of progression between infection stages;
    $\delta$: partnership duration.}
\end{figure}
%===================================================================================================
\subsection{Equations}\label{foi.prop.eq}
Since partnership duration is now considered separately and explicitly,
we do not define any per-partnership probability of transmission $B$.
Rather, we define the force of infection to directly include
the frequency of sex per partnership $F$ and probability of transmission per sex act $\beta$.
However, the mixing is slightly more complicated than before,
since the number of ``effective partnerships'' depends on infection status.
In addition, these partnerships are now defined as numbers of current partners~$K$,
rather than rates of partnership formation~$Q$.
\par
Let $M_{pii'}$ be the total (population-level, not per-person)
number of type-$p$ partnerships between risk groups $i$~and~$i'$.
This ``mixing matrix'' $M_{pii'}$ can be defined in several ways (\eg \sref{mod.par.mix}),
based on the total numbers of ``effective partnerships'' among each group: $M_{pi}, M_{pi'}$,
plus some parameter(s) specifying mixing patterns.
Working backwards, we start by defining $M_{pi}$ (and likewise $M_{pi'}$) via
the sum across health statuses --- \ie susceptible and different stages of infection~$h$:
\begin{equation}\label{eq:M.SI}
  M_{pi} = M_{S,pi} + \sum_h M_{I,pih}
\end{equation}
We define the total numbers of partnerships among susceptible individuals as:
\begin{equation}\label{eq:M.S}
  M_{S,pi} = S_{i} K_{pi}
\end{equation}
and likewise for individuals in infection stage $h$ as:
\begin{equation}\label{eq:M.I}
  M_{I,pih} = I_{ih,\p=p} (K_{pi}-1) + \sum\nolimits_{\p \ne p} I_{ih\p}\,K_{pi}
\end{equation}
This \eqref{eq:M.I} is the key equation whereby
the  numbers of ``effective type-$p$ partnerships'' among
individuals in stratum $\p$ are reduced by 1.
This reduction is then propagated through the mixing patterns when defining $M_{pii'}$.
% TODO: (*) double-check that this reduction should be propagated through the mixing patterns
Next, we define the total (population-level, not per-person) rate of transmission
from group $i'$ and infection stage $h'$ to group $i$ via type-$p$ partnerships as:
\begin{equation}
  \Lambda_{pii'h'} = F_p \beta_{pii'h'} M_{pii'}
  \left(\frac{M_{S,pi}}{M_{pi}}\right)
  \left(\frac{M_{I,pi'h'}}{M_{pi'}}\right)
\end{equation}
where the two fractions represent the proportions of all type-$p$ partnerships $M_{pii'}$
that are formed between susceptible individuals from group $i$ ($M_{S,pi}$)
and infectious individuals in group/infection stage $i'h'$ ($M_{I,pi'h'}$).
The per-person transmission rates to group $i$, and from group $i'h'$
(Figure~\ref{fig:model.hiv.p}) can then be defined as:
\begin{alignat}{1}
  \lambda_{pi} &= \sum_{i'h'} \Lambda_{pii'h'}\,{S_{i}}^{-1} \label{eq:foi.i} \\
  \lambda'_{pi'h'} &= \sum_{i} \Lambda_{pii'h'}\,{I_{i'h'}}^{-1} \label{eq:foi.jh}
\end{alignat}
For the purposes of solving the model,
we can skip division by $S_{i}$ and $I_{i'h'}$ in \eqrefs{eq:foi.i}{eq:foi.jh},
since $\lambda'_{pi}$ and $\lambda'_{pi'h'}$ are immediately multiplied by $S_{i}$ and $I_{i'h'}$,
respectively, in the system of differential equations
--- \ie we need total, not per-person, rates of transmission.
\par
Finally, and to reiterate from above, infected individuals in stratum $I_{ih\p}$
are assumed to form new partnerships at a rate $\delta_p^{-1}$,
and thereby transition to stratum $I_{ih\p_0}$ (``all new partners''); and
otherwise transition between infection stages, cascade of care, activity groups, etc. as usual,
as illustrated in Figure~\ref{fig:model.hiv.p}.
%===================================================================================================
\subsection{Transmission via Multiple Partnerships}\label{foi.prop.mp}
In the proposed \emph{Effective Partnerships Adjustment} approach,
we do not explicitly model the proportions of infected individuals
who recently acquired and/or transmitted infection via
two partnerships of the same type or two different partnership types.
To do so, the required size of the new dimension $\p$ would be at least $2^{P}$, not $P+1$,
where $P$ is the number of different partnership types modelled.
For transmission via three different partnerships,
the required size would be at least $3^{P}$, and so on.
Indeed, this exponential relationship is related to the challenge of specifying
all possible combinations of partnership states in pair-based models \cite{Kretzschmar2017}.
However, under frequentist assumptions, we can equivalently model
two transmissions by one individual as one transmission each by two individuals.
Thus, we can transfer two individuals from $I_{ih\p_0}$ to
$I_{ih\p_1}$ and $I_{ih\p_2}$ (one each) under the proposed $P+1$ stratification,
instead of just one individual from $I_{ih\p_0}$ to
``$I_{ih\p_{12}}$'' under an exponential ($x^P$) stratification.
\par
In fact, $I_{ih\p_0}$ can be \emph{negative} (but only for $\p = 0$),
because the dimension $\p$ is only relevant to \eqref{eq:M.I};
in all other contexts and equations,
we use $I_{ih} = \sum_{\p} I_{ih\p}$, which must be positive as usual.
Moreover, we can also have $I_{ih\p} > I_{ih}$, provided that:
\begin{equation}\label{eq:I.constr}
  I_{ih\p} \le I_{ih} K_{pi}
\end{equation}
reflecting the situation where $>100$\% of $I_{ih}$
have recently acquired and/or transmitted infection via at least one type-$p$ partnership,
or $>50$\% via at least two partnerships, etc.
This situation can therefore only arise in the context of
multiple concurrent type-$p$ partnerships: $K_{pi} > 1$.
If $I_{ih\p} > I_{ih}$, then $I_{ih\p_0}$ \emph{must} be negative,
but we can show that \eqref{eq:M.I} still yields the correct value of $M_{I,pih}$.
With this perspective, the constraint in \eqref{eq:I.constr} may be more intuitive:
we cannot ``remove'' more than the total number of current partnerships.
This constraint should also be easy to guarantee for a small enough timesteps,
because in \eqref{eq:M.I}, $M_{I,pih}$ approaches zero as $I_{ih\p}$ approaches $I_{ih} K_{pi}$
--- i.e. all type-$p$ partnerships become seroconcordant-infected,
and no more transmission can occur via these partnerships until partners change.
%===================================================================================================
\subsection{Transmission-Driven Seroconcordance}\label{foi.prop.tdsc}
Another benefit of the proposed approach is that
we can quantify the proportion of partnerships that are seroconcordant
due to prior transmission within the partnership.
We call these partnerships ``transmission-driven seroconcordant''
to distinguish them from seroconcordant partnerships \emph{newly} formed
by chance or due to serosorting \cite{Purcell2017,Kim2020}
among two previously infected individuals.
For this proportion,
the numerator is the population size of stratum $\p=p$, while
the denominator is the total number of type-$p$ partnerships among
(a) all individuals, or (b) infected individuals only:
\begin{equation}\label{eq:tdsc}
  \textsc{tdsc}_{p*} = \frac{I_{*,\p=p}}{K_{p*} \sum_{\p} X_{*\p}}
  \qquad
  \begin{cases}
    X = I + S & \textnormal{(a)} \\
    X = I     & \textnormal{(b)} \\
  \end{cases}
\end{equation}
where $*$ could specify any subset of the population
defined by modelled strata --- \eg sex, activity group, HIV state, etc.
While denominator (a) may be interesting conceptually,
denominator (b) can be more directly interpreted as
a relative reduction in onward transmission risk among infected individuals
due to infections ``trapped'' within partnerships.
The actual total reduction would be weighted by
the probability of transmission per sex act, sex frequency, etc.
We also note that a single post-transmission partnership
will be ``double-counted'' in the numerator of \eqref{eq:tdsc},
because both infected individuals cannot transmit via this partnership.
