\section{Introduction}\label{intro}
In compartmental models of infectious disease transmission,
the ``force of infection'' equation defines
the rate at which susceptible individuals acquire infection.
This equation therefore mechanistically integrates various relevant factors, such as
numbers and types of contacts per person,
the probability of transmission per contact,
health states, and interventions,
each of which may vary across a population and/or time.
In the context of HIV and other sexually transmitted infections,
these contacts specifically represent ``sex acts'',
which often occur within persistent sexual partnerships.
Thus, sexual partnership dynamics are key factors in
the force of infection equation for sexual transmission of HIV \cite{Rao2021}.
\par
During our previous review of compartmental models of HIV transmission \cite{Knight2022sr},
we noted several different approaches to integrating sexual partnerships
within HIV force of infection equations.
These equations differed not only in which
risk groups, partnership types, health states, and/or interventions were modelled,
but also in which mathematical approximations of sexual partnership dynamics were used
(see also \cite{Rao2021}).
Previous work comparing different modelling frameworks
--- \ie compartmental \vs pair-formation \vs individual-based models ---
has shown that similar differences can strongly influence key model outputs,
such as inferred parameter values and projected intervention impacts
\cite{Kretzschmar1998,Eames2002,Lloyd-Smith2004,Johnson2016mf}.
However, no study has examined the specific differences that we identified
among force of infection equations in compartmental models of HIV transmission.
% TODO: (?) comment on unique timescale of HIV
\par
Therefore, regarding different approaches to modelling HIV transmission via sexual partnerships,
we sought to:
critically review assumptions and limitations of prior approaches (\sref{foi.prior}),
propose a new approach which overcomes several limitations of prior approaches (\sref{foi.prop}),
and compare key model outputs under prior \vs proposed approaches (\sref{foi.exp}).
%---------------------------------------------------------------------------------------------------
\paragraph{First Principles \& Post-Transmission Contacts}
In the simplest compartmental transmission models,
the force of infection $\lambda$ can be defined as:
\begin{equation}\label{eq:foi.simple}
  \lambda = C \beta \frac{I}{N}
\end{equation} where:
$C$ is the average contact rate per-person;
$\beta$ is the average probability of transmission per contact; and
$I/N$ is the current prevalence of infection.%
\footnote{\eqref{eq:foi.simple} also assumes
  frequency-dependent transmission \vs density-dependent transmission,
  which is almost always more appropriate for sexually-transmitted diseases \cite{Begon2002}.}
\par
If the population is stratified into multiple groups $i$,
the infection is stratified into multiple infectious stages $h$,
and contacts are stratified into multiple types $p$,
then \eqref{eq:foi.simple} can be generalized to:
\begin{equation}\label{eq:foi.strat}
  \lambda_i = \sum_{pi'h'} C_{pii'} \beta_{ph'} \frac{I_{i'h'}}{N_{i'}}
\end{equation}
where:
$C_{pii'}$ is the average rate of type-$p$ contacts per-person among group $i$ with group $i'$,
$\beta_{ph'}$ is the average probability of transmission per type-$p$ contact given infection stage $h'$,
and $I_{i'h'}/N_{i'}$ is the prevalence of infection stage $h'$ among group $i'$.
Note that \eqref{eq:foi.strat} implicitly assumes that
contact rate and mixing by infection status/stage is random.
\par
The force of infection equation is further complicated by
repeated contacts with the same individuals, such as in sexual partnerships
(also household contacts, and other social relationships),
where each contact reflects a single sex act.
With repeated \vs random contacts, individuals who recently acquired or transmitted infection
will continue to contact the same person, resulting in ``post-transmission contacts'' (PTC)
--- sometimes called ``wasted contacts'' (in terms of transmission) ---
and slower infection spread through the contact/partnership network \vs without PTC.%
\footnote{Another conception of ``post-transmission contacts'' is HIV seroconcordance.}
A major focus when modelling HIV transmission via sexual partnerships
is therefore to account for these PTC.
