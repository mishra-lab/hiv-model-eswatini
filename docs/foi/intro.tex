\section{Introduction}\label{intro}
% JK: introduce the subject
The dynamics of sexually transmitted infections (STIs) are rooted in
complex patterns of sexual partnerships formation and dissolution.
Such patterns should therefore be reflected in mathematical models of STI transmission
--- in particular, within the force of infection (incidence) equation.
\par
% JK: describing the problem
In classic compartmental models (CCMs), sexual partnerships are difficult to model
because model variables represent homogeneous groups of people,
whose individual partnerships cannot be tracked.
In such models, partnerships are modelled as ``instantaneous'',
such that the cumulative risk of transmission per-partnership is applied
to the average proportion of people forming new partnerships per unit time.
In CCMs, newly infected people are therefore modelled to be
immediately at risk of onward transmission,
and thus, infection incidence is proportional to prevalence.
As such, CCMs have several limitations, including:
attributing too much transmission to longer duration partnerships and acute HIV infection,
and failure to accurately reflect partnership concurrency \cite{Johnson2016a,Rao2021}.
\par
% JK: other solutions
The limitations of CCMs in capturing partnership dynamics
have motivated the use of other modelling frameworks for STIs,
including pair-based models and individual-based models \cite{Rao2021}.
Pair-based models are an extension of CCMs in which
model variables represent homogeneous groups of people in various partnership states
\cite{Dietz1988,Ferguson2000,Kretzschmar2017}, while
individual-based models represent unique people and their partnerships explicitly
\cite{vanImhoff1998,Bershteyn2013}.
In pair-based models, the number of variables (compartments) required can grow exponentially with
the modelled numbers of risk groups and concurrent partners per-person,
often rendering such models impractical \cite{Rao2021}.
Similarly, individual-based models typically require more data and analysis to parameterize
than compartmental models, and can be computationally expensive to run \cite{Rao2021}.
so, CCMs are still used despite their limitations,
especially for slower STIs like HIV \cite{Rao2021}.
% JK: \citet{Garnett1996} briefly explored an extension related to the proposed work...
%     not sure where to mention, maybe discussion?
\par
% JK: sign-post the paper
In this work, we propose a new force of infection model within the CCM framework,
that captures population-level dynamics of partnership duration and concurrency.
First (\sref{prior}), we review existing models of the force of infection in CCMs
and their inherent assumptions.
Next (\sref{prop}), we develop the proposed model.
Then (\sref{exp}), we integrate the proposed model
into an existing model of heterosexual HIV transmission in Eswatini,
and compare key model outputs under the proposed vs existing models.
We conclude (\sref{disc}) with some discussion of the results and future work.
