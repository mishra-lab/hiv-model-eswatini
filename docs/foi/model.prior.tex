\section{Prior Approaches: Instantaneous Partnerships}\label{foi.prior}
The earliest HIV transmission models \cite{Anderson1986}
were adapted from models of other sexually transmitted diseases,
especially gonorrhea \cite{Yorke1978,Nold1980,Hethcote1982}.
These early HIV transmission models did not explicitly model individual sex acts,
but instead assumed an overall probability of transmission per partnership \cite{Isham1988}.
This assumption was initially justified via data suggesting that
the probability of HIV transmission within partnerships
increased quickly and then saturated \cite{Kaplan1990}.
Such data were later explained by heterogeneity in
infectiousness (\eg due to infection stage, etc.) and/or
susceptibility (\eg due to genital ulcer disease, etc.)
\cite{Gray2001,Rottingen2002,Boily2009}.
As this heterogeneity was quantified \cite{Gray2001}
and incorporated into HIV transmission models \cite{Moghadas2003},
the probability of transmission was increasingly parameterized per act \vs per partnership.
\par
The shift to per-act \vs per-partnership parameterization highlighted
a fundamental limitation of compartmental models:
compartmental models cannot model individual partnerships,
because each ``compartment'' reflects a group of individuals
which is assumed to be homogeneous and memoryless \cite{Rao2021}.
Therefore, the dynamics of sexual partnerships must be modelled using
average rates of partnership change and average characteristics of those partnerships.
As a result, partnerships are effectively modelled as instantaneous, such that
the cumulative risk of transmission per partnership
is applied at the moment of partnership change \cite{Dietz1988sti}.
This cumulative risk can be defined in terms of
the average total numbers of sex acts per partnership,
but the timing of specific sex acts or other events within partnerships
cannot be captured in compartmental models.
Further implications of the ``instantaneous partnership assumption''
and alternate modelling frameworks which avoid this assumption
are discussed below in \sref{foi.prior.lims}, \ref{foi.prior.alt}, and \ref{foi.prop}.
\par
Thus, over the years, different force of infection equations
have been designed for compartmental models which
explicitly aggregate the risk of transmission across different numbers and types of sex acts,
and likewise across different numbers and types of sexual partnerships.
The remainder of this section reviews these equations and their assumptions in detail.
%===================================================================================================
\subsection{Aggregating Sex Acts within a Partnership}\label{foi.prior.bhom}
To account for inert sex acts within ongoing partnerships,
the per-partnership probability of transmission $B$ was conceptualized as follows \cite{Allard1990}.
Let $A$ denote the total number of sex acts in the partnership,
and $\beta$ denote the probability of transmission per act.
For now, $\beta$ is assumed to be equal (constant) for all acts.
With equal $\beta$, the theoretical probability of $n$ transmissions after $A$ acts
can be described by a binomial distribution:
\begin{equation}\label{eq:B.n}
  p(n) = {A \choose n}\,\beta^n\,{(1 - \beta)}^{A-n}
\end{equation}
Since transmission of HIV can actually only occur once,
the per-partnership probability of transmission $B$ is defined via
the probability of ``escaping'' infection after all $A$ acts:%
\footnote{\eqref{eq:B} can also be reasonably approximated
  via the Poisson distribution $B = 1 - e^{-\beta A}$ for small $\beta$.}
\begin{alignat}{1}\label{eq:B}
  B &= 1 - p(n = 0) \nonumber\\
  &= 1 - {A \choose 0}\,\beta^0\,{(1 - \beta)}^{A} \nonumber\\
  &= 1 - {(1 - \beta)}^A
\end{alignat}
Although $B(A)$ is monotonic increasing,
the effective probability of transmission per act $B/A$
decreases as the number of acts $A$ increases because,
on average, more and more acts are inert --- \ie occur after transmission.
Figure~\ref{fig:binom.dur} illustrates the shape of $B(A)$ (red)
and the corresponding effective probabilities of transmission per act $B/A$ (tangent slopes)
for a shorter (blue, fewer sex acts) \vs longer (green, more sex acts) partnership.
The expected proportion of inert sex acts in a partnership
also increases with the per-act probability of transmission $\beta$,
as follows (illustrated in Figure~\ref{fig:binom.inert}):
\begin{equation}\label{eq:inert}
  P_{\txn{inert}} = 1 - \frac{B}{\beta A}
\end{equation}
\begin{figure}
  \centerline{
  \begin{subfigure}[b]{.5\linewidth}
    \centering\includegraphics[scale=1]{binom.dur}
    \caption{Probability of transmission per partnership $B$ \vs number of sex acts $A$,
      comparing shorter (blue) \vs longer (green) partnerships}
    \label{fig:binom.dur}
  \end{subfigure}\hfill
  \begin{subfigure}[b]{.475\linewidth}
    \centering\includegraphics[scale=1]{binom.dots.xph}
    \caption{Average accumulation of transmission probability for
      within-partnership heterogeneity (left) \vs
      between-partnership heterogeneity (right)}
    \label{fig:binom.xph}
  \end{subfigure}}
  \caption{Per-partnership probability of transmission \vs number of acts}
  \label{fig:binom}
  \floatfoot{
    $B$: probability of transmission per partnership;
    $\beta$: probability of transmission per act;
    $A$: total acts per partnership;
    $\alpha$: fraction of total acts (within or between partnerships).}
\end{figure}
%===================================================================================================
\subsection{Heterogeneity in the Per-Act Probability of Transmission}\label{foi.prior.xph}
As noted above, the per-act probability of transmission $\beta$ is actually heterogeneous,
varying with factors like: HIV infection stage, genital ulcer disease, condom use, etc.
\cite{Boily2009,Giannou2016}.
The next step in developing a force of infection equation is to extend \eqref{eq:B}
to allow heterogeneity in $\beta$ \cite{Allard1990}.
Let $\beta_f$ denote the probability of transmission associated
with a particular factor (or combination of factors) $f$; and
let $\alpha_f$ denote the proportion of acts $A$ in an average partnership
having transmission probability $\beta_f$
(thus $\Sigma_f \alpha_f = 1$).
There are two main approaches to aggregating $\beta_f$,
reflecting different interpretations of $\alpha_f$:%
\footnote{In most compartmental models without repeated contacts (partnerships),
  this distinction is not possible or necessary, because
  all contacts (sex acts) between two compartments (risk groups) are assumed to be independent.}
\begin{itemize}
  \item \textbf{{Within}-Partnership Heterogeneity (WPH)}:
  modelled partnerships are identical, but comprise heterogeneous acts
  --- $\alpha_f$ denotes a proportion of acts in each partnership
  (Figure~\ref{fig:binom.xph} left).
  % \emph{some acts in all partnerships}
  \begin{equation}\label{eq:B.wph}
    B_{\wph} = 1 - \prod_f {(1 - \beta_f)}^{A\alpha_f}
  \end{equation}
  \item \textbf{{Between}-Partnership Heterogeneity (BPH)}:
  modelled partnerships are different, but each comprise identical acts
  --- $\alpha_f$ denotes a proportion of partnerships
  (Figure~\ref{fig:binom.xph} right).
  % \emph{all acts in some partnerships}
  \begin{equation}\label{eq:B.bph}
    B_{\bph} = 1 - \sum_f \alpha_f {(1 - \beta_f)}^{A}
  \end{equation}
\end{itemize}
Figure~\ref{fig:binom.xph} illustrates these approaches for a simple case with two factors.
For WPH (left), each factor $f$ marginally contributes to
the probability of escaping infection in every partnership.
For BPH (right), the overall probability of escaping infection is modelled as
a weighted average across partnerships, each affected by a single factor $f$.
Both approaches guarantee $B < 1$,
but we can show that $B_{\wph} \ge B_{\bph}$
by the (weighted) AM-GM inequality (see \sref{sr.foi.proof}) \cite{Aldaz2009}.
Intuitively, this inequality arises because
any large probability of transmission $\beta_f$
has disproportionate influence in \eqref{eq:B.wph},
even for a small proportion of acts affected $\alpha_f$,
whereas this influence is bounded by $\alpha_f$ in \eqref{eq:B.bph},
as shown in Figure~\ref{fig:binom.xph}.
\par
The decision to use WPH \vs BPH for aggregating specific types of heterogeneity in $\beta$
should be driven by the factor(s) in question.
To this end, it is possible to combine \eqref{eq:B.wph} and \eqref{eq:B.bph} as follows
to aggregate both types of factors simultaneously:
\begin{equation}\label{eq:B.xph}
  B_{\xph} = 1 - \sum_g \gamma_g \prod_f {(1 - \beta_{fg})}^{A\alpha_{fg}}
\end{equation} where:
$f$ denotes WPH factor(s);
$g$ denotes BPH factor(s); and
$\gamma_g$ replaces $\alpha_f$ for BPH factors.
Then, for example, if it is known or assumed that
``50\% condom use'' reflects 50\% condom use in 100\% of partnerships,
sex acts with condoms \vs without condoms
should be aggregated as WPH, with $\alpha_f = 0.5$.
By contrast, heterogeneity in individual-level factors like infection stage or treatment status
should be aggregated as BPH,%
\footnote{\label{foot:xph.future}
  Individual-level factors should be aggregated as BPH because
  a given partner has exactly one current infection stage or treatment status;
  of course, this stage/status could evolve over the course the partnership,
  but this future trajectory is not explicitly modelled
  --- which only serves to highlight the limitations of
  either approach to aggregating heterogeneity in $\beta$.}
with $\gamma_g$ as the conditional prevalence of each stage/status~$g$ among infected partners.
In fact, aggregating infection stage and treatment status is often deferred to
the full incidence equation (see \sref{foi.prior.part}) using an equivalent form,
but where $\gamma_g$ is replaced by
the unconditional prevalence of stage/status~$g$ among \emph{all} partners.
%===================================================================================================
\subsection{Aggregating Partnerships}\label{foi.prior.part}
Although we considered between-partnership heterogeneity in \sref{foi.prior.xph},
the modelled per-partnership probability of transmission $B$
still corresponds to a single average partnership.
Some population groups may have multiple partners per unit time (usually year),
possibly including different types of partnerships,
or less than one partnership per year, on average.
Thus, the second step in constructing the incidence equation is to
aggregate transmission risk across these various partnerships / exposures \cite{Allard1990}.
\par
As in \sref{foi.prior.xph}, there are two main approaches to aggregating partnerships
--- indeed having similar equations to \eqrefs{eq:B.wph}{eq:B.bph}:
\begin{itemize}
  \item \textbf{Incidence Rate:}
  instantaneous rate of infection among susceptible individuals
  --- transmission risks are additive; can have $\lambda_i^\ir > 1$.
  \begin{equation}\label{eq:foi.ir}
    \lambda_i^\ir = \sum_{pi'h'} Q_{pii'} B_{pii'h'} \frac{{I}_{i'h'}}{N_{i'}}
  \end{equation}
  \item \textbf{Incidence Proportion:}
  cumulative proportion of susceptible individuals infected over a period $\Delta_t$
  --- transmission risks are competing; can only have $\lambda_i^\ip \le 1$.
  \begin{equation}\label{eq:foi.ip}
    \lambda_i^\ip = 1 - \prod_{pi'h'} {\left(1 - B_{pii'h'} \frac{{I}_{i'h'}}{N_{i'}}\right)}^{Q_{pii'} \Delta_t}
  \end{equation}
  % TODO: (?) describe why this looks like ~ wph even though partnerships are 'distinct'
  % should *not* aggregate as if all partnerships are same i'h'
\end{itemize} where:
$Q_{pii'}$ is the rate of type-$p$ partnership formation between groups $i$ and $i'$,%
\footnote{This matrix $Q_{pii'}$ is often broken down into
  an overall partnership formation rate $Q_{pi}$ and a mixing matrix $\rho_{pii'}$.}
$B_{pii'h'}$ is the average per-partnership probability of transmission
from group/infection stage $i'h'$ to group $i'$ via partnership type $p$,
and $I_{i'h'}/N_{i'}$ is the prevalence of infection stage $h'$ among group $i'$.
Similar to within- \vs between-partnership heterogeneity,
we can show that $\lambda^\ir \ge \lambda^\ip$ (see \sref{sr.foi.proof}).
\par
The force of infection is a rate by definition \cite{Anderson1991}.
Yet, in principle, incidence proportion could be more precise than incidence rate
\emph{over a given time period $\Delta_t$}.
Since most models are now solved computationally,
this period $\Delta_t$ could be matched to the timestep of the numerical solver.%
\footnote{\label{foot:ode}%
  Popular numerical solvers include:
  \href{https://docs.scipy.org/doc/scipy/reference/generated/scipy.integrate.odeint.html}
  {\texttt{scipy.integrate.odeint}} in Python,
  \href{https://cran.r-project.org/web/packages/deSolve/index.html}
  {\texttt{deSolve::lsoda}} in R, and
  \href{https://www.mathworks.com/help/matlab/ref/ode45.html}
  {\texttt{ode45}} in MATLAB.
  These solvers can use adaptive timesteps for precision,
  but only pass the current time $t$, not the timestep $\Delta_t$, to the derivative computing function.}
However, the added precision may be insignificant, because such timesteps should already be small.%
\footnote{If the timestep must remain large due to computational constraints, then
  modellers should consider whether any other rates, especially large rates,
  should be similarly adjusted for the timestep.}
Moreover, some applications of incidence proportion have used
a period of $\Delta_t = 1$ year in the equation, but then
applied the result as a rate over smaller timesteps.
Such applications erroneously reduce transmission within each \emph{current} timestep
in anticipation of competing risks between partnerships across \emph{future} timesteps.
These competing risks are already captured via
loss of susceptibles to infection over successive timesteps.
For example, if we compute $\lambda^\ip = 0.2$ using $\Delta_t = 1$ year,
but we apply this incidence using a timestep of 1 month,
then the newly infected proportion after 1 year would be modelled as:
$1 - {(1 - \lambda^\ip/12)}^{12} = 0.18 < 0.20$.
While $\Delta_t = 1$ year may be chosen to match common reporting periods for sexual behaviour data,
this choice remains mathematically arbitrary,
and often coincides with ignoring partnership dynamics beyond 1 year,
as discussed below in \sref{foi.prior.dur}.
In conclusion, unless the period $\Delta_t$ can be matched to the numerical solver timestep,
incidence rate \eqref{eq:foi.ir} should be preferred over incidence proportion \eqref{eq:foi.ip}.
% cite Diekmann2021 ?
%===================================================================================================
\subsection{Revisiting Partnership Duration}\label{foi.prior.dur}
A final decision in constructing the force of infection equation relates to parameterization.
In \eqrefr{eq:B.n}{eq:foi.ip}, partnership durations $\delta$ are not explicitly modelled,
but implied by the total numbers of sex acts per partnership $A$,
and a presumed frequency of sex per partnership $F$, such that $A = F\delta$.
By contrast, the partnership formation rate $Q$ is often directly informed by survey questions like
\shortquote{How many different people have you had sex with in the past 12 months?}
As such, the lowest possible value among sexually active individuals
could naively be taken as $Q = 1$ (per year).
Then, if $Q \ge 1$ is used in the model,
the total sex acts per partnership can (and should) be reduced to reflect up to one year
--- \ie $A \le F$, or effectively $\delta \le 1$ year.
\par
Using $Q \ge 1$ and $A \le F$
can overestimate transmission via longer ($\delta > 1$ year) partnerships,
because the proportion of inert sex acts is effectively underestimated.
On the other hand, using the true $Q < 1$ and $A > F$
can delay transmission in longer partnerships
because inert sex acts which truly accumulate later in partnerships
are modelled as an average proportion throughout the partnership
(\ie tangents \vs curve in Figure~\ref{fig:binom}).
These dynamics are further explored in simulation experiments (\sref{foi.exp.model}).
\par
Lastly, we note that partnership duration $\delta$ is further related to
the average partnership formation rate $Q$ and
the average number of concurrent partners $K$ by $Q = K/\delta$.
% \footnote{Gaps between partnerships do not result in $Q < K/\delta$,
%   because the average $K$ would be reduced if some individuals are ``between partnerships''.}
Thus, an alternate force of infection parameterization could specify
the number of concurrent partners $K$ and the frequency of sex with each partner $F$.
The overall rate of sex would be the same: $QA = KF$, since $A = F\delta$.
In some ways, this $KF$ parameterization is more intuitive,
and it will be useful in our new force of infection approach (\sref{foi.prop}).
%===================================================================================================
\subsection{Limitations of Prior Approaches}\label{foi.prior.lims}
The limitations of the above approaches to modelling HIV transmission via sexual partnerships,
along with their implications for existing model-based evidence,
can be summarized as follows (see also \cite{Rao2021}).
%---------------------------------------------------------------------------------------------------
\paragraph{Instantaneous Partnerships}
\eqrefs{eq:foi.ir}{eq:foi.ip} both include
the current HIV prevalence $I/N$ directly in the force of infection.
Thus, newly infected individuals are modelled to be
immediately at risk of onward transmission,
including via the exact same partnership by which they were infected.
Similarly, individuals who recently transmitted to a given partner are also modelled to be
at risk of transmitting (again) to the same partner,
albeit with a small absolute rate reduction due to the smaller susceptible population.
This modelling assumption acts to increase the modelled rate of transmission \vs reality,
especially for longer partnerships.
As a result, the contribution of longer partnerships to overall transmission could be overestimated,
while the contribution of shorter partnerships could be underestimated.
%---------------------------------------------------------------------------------------------------
\paragraph{Aggregating Past/Future Sex Acts}
The instantaneous partnerships assumption is in fact directly related to inert sex acts,
because the delay in onward transmission risk that is missing under instantaneous partnerships
reflects the same post-transmission period within partnerships
wherein additional sex acts cannot result in more infections.
The prevailing solution to this issue is to
define the per-partnership probability of transmission $B$ by
aggregating competing risks from each sex act within a given partnership via \eqref{eq:B} et al.
However, as described in \sref{foi.prior.dur}, this approach introduces a trade off between
capturing the true proportion of inert sex acts in longer partnerships
(using the true partnership duration~$\delta$)
\vs capturing the true magnitude of early transmission within partnerships (using $\delta \le 1$).
These two options would then underestimate or overestimate
the contribution of longer partnerships to overall transmission, respectively.
Moreover, the sex acts aggregated within each partnership via \eqref{eq:B} et al.
are almost always parameterized to reflect current conditions
--- \ie HIV stage, treatment status, condom use, etc. ---
even though such conditions evolve over the course of partnerships,
especially longer partnerships.%
\footnote{Again, we can see the connection to the instantaneous partnerships assumption.
  This issue also parallels limitations of cross-sectional HIV risk factor analyses,
  where risk factors are modelled as static,
  but true risk accumulates via cumulative exposure to dynamic risk factors.}
The implications of aggregating these past/future sex acts are not immediately obvious,
and likely depend on numerous factors and conditions.
%---------------------------------------------------------------------------------------------------
\paragraph{Incidence Proportion}
Risk from multiple partnerships is sometimes aggregated as
incidence proportion $\lambda^\ip$ via \eqref{eq:foi.ip}.
As noted in \sref{foi.prior.part}, this approach is not inherently wrong,
but the specified time period $\Delta_t = 1$ often is.
This $\Delta_t$ \emph{should} be matched to the timestep of the numerical solver,
but $\Delta_t = 1$ year is often used,
and the resulting incidence applied as a rate over smaller timesteps, reducing transmission.
Since $\lambda^\ip$ saturates at 1 --- similar to $B(A)$ in Figure~\ref{fig:binom.dur} ---
transmission to higher risk groups is disproportionately reduced.
%---------------------------------------------------------------------------------------------------
\paragraph{Within \vs Between Partnership Heterogeneity}
A final limitation of prior approaches is the apparent lack of distinction between
within- \vs between-partnership heterogeneity when computing
the average per-partnership probability of transmission $B$.
Both WPH and BPH --- \ie \eqrefs{eq:B.wph}{eq:B.bph} --- and combinations thereof, have been used
to model modified transmission risk in a proportion of sex acts due to
HIV stage, treatment status, PrEP use, condom use, STI co-infection, circumcision, and more,
but the choice of aggregation model is almost never explicitly justified.
For some factors, there may be no ``correct'' choice,
but modellers should be aware of the assumptions implied by their choice.
The implications of model choice for transmission dynamics
mainly derive from the fact that $B_{\wph} \ge B_{\bph}$,
but even then differences are often small (see \sref{foi.exp.xph}). % TODO: (!)
%===================================================================================================
\subsection{Alternate Modelling Frameworks}\label{foi.prior.alt}
Recognizing the limitations of compartmental models in simulating
infectious disease transmission via sexual partnerships,
two main alternate modelling frameworks have been developed \cite{Rao2021}.
These frameworks are illustrated in Figure~\ref{fig:fw}.
Modelling frameworks are further classified Appendix~1 of \cite{Johnson2016mf}.
\begin{figure}
  \centering\foreach \var/\cap in
  {comp/Classic compartmental,pair/Pair-based,ibm/Individual-based}{%
  \begin{subfigure}[b]{.33\linewidth}
    \centering
    \includegraphics[width=.9\linewidth]{mf.\var}
    \caption{\cap}
    \label{fig:mf.\var}
  \end{subfigure}}
  \caption{Representations of health states and sexual partnerships
    under 3 different STI modelling frameworks}
  \label{fig:fw}
  \floatfoot{
    $S$: susceptible; $I$: infectious;
    solid arrows: partnerships; dashed arrows: state transitions.}
\end{figure}
%---------------------------------------------------------------------------------------------------
\paragraph{Pair-Based Models}
Pair-based models, also known as pair-formation models, were developed as early as 1988,
with the explicit motivation to overcome limitations of
classic compartmental models of STI transmission \cite{Dietz1988sti}.
In pair-based models, the fundamental population stratification
reflects different partnership configurations and health states \cite{Kretzschmar2017}, such as:
susceptible and single, a susceptible/infected pair in a long-term partnership, etc.
(Figure~\ref{fig:mf.pair}).
Such models can therefore track the numbers of partnerships
where transmission is \vs is not possible,
thereby avoiding the instantaneous partnership assumption.
Pair-based models have been applied to a variety of STIs \cite{Kretzschmar2017}.
However, the numbers of compartments required to reflect all possible partnership configurations
\emph{and} all possible health states among connected partners
quickly become impractical \cite{Kretzschmar2017,Rao2021}.
For example, a classic compartmental model with 2 risk groups and 2 health states
would require $2\times2=4$ compartments;
whereas even a first-order pair-based model (\ie without ``triples'')
would require $2\times2$ (singles) + ${(2\times2)}^2$ (pairs) ${} = 20$ compartments;
a second-order pair-based model (\ie with ``tripples'')
would require $4 + 4^2 + 4^3 = 84$ compartments.
Thus, pair-based models are especially limited in their ability to model partnership concurrency
--- the role of which in HIV epidemiology remains controversial \cite{Sawers2013}.
As such, pair-based models have seen little widespread adoption
for HIV transmission modelling~\cite{Rao2021}.
%---------------------------------------------------------------------------------------------------
\paragraph{Individual-Based Models}
Individual-based models, also knowns as agent-based, network-based, or microsimulation models,
explicitly simulate unique individuals (Figure~\ref{fig:mf.ibm}).
They represent a fundamental change in the model unit from groups of individuals
--- \ie the ``compartments'' of compartmental models \cite{Rao2021}.
Individual-based models can therefore model unique partnerships, and track them over time.
Such individuals and partnerships can then be
parameterized in fundamentally different ways \vs compartmental models,
including with continuous valued features like infection age and sex frequency,
\vs predetermined categories like infection stages and partnership types \cite{Pellis2015,Rao2021}.
Parameters for each individual and partnership are thus sampled randomly and/or dynamically,
allowing more complete and nuanced representations of risk heterogeneity and partnership dynamics.
Such nuances can in fact be key determinants of epidemic dynamics and intervention impact
\cite{Hontelez2013,Johnson2016mf}.
Evidently, many of the limitations of compartmental and pair-based models
do not apply to individual-based models \cite{Rao2021}.
Yet these limitations are replaced with new challenges,
especially related to implementing, parameterizing, and calibrating these powerful models
\cite{Rao2021,Pellis2015,Hazelbag2020}.
For example, much effort has been dedicated to
formalizing the statistical properties of dynamic networks
via temporal exponential family random graph models (tERGM) \cite{Jenness2018}
or latent order logistic models (LOLOG) \cite{Clark2022},
so that dynamic networks can be generated which are consistent with observed data.
Although individual-based models have seen greater use than pair-based models,
these challenges still prevent universal adoption over classic compartmental models \cite{Rao2021}.
It's worth noting that not all individual-based models are transmission models,
as individual-based models can also be used for
simulation and inference for non-infectious diseases~\cite{Silverman2021}.
