In classic compartmental models of HIV and other sexually transmitted infections,
populations are stratified into homogeneous, memoryless states
such that timing of transmission within specific sexual partnerships cannot be tracked.
Instead, these models define the force of infection (incidence rate) via
mean rates of partnership change, multiplied by
cumulative probabilities of transmission per partnership (or partnership-year).
Thus, partnerships are effectively modelled as instantaneous.
In this paper, we critically review
the assumptions and limitations of this ``instantaneous partnerships'' approach,
including different variations thereof.
We then propose a new approach, the Effective Partnerships Adjustment (EPA),
which overcomes several of these limitations.
EPA adds a new population stratification to track
individuals in partnerships where transmission has already occurred,
and then reduces these individuals' effective partnerships by one,
until they change partners.
Unlike pair-based models, EPA only adds one stratification per partnership type,
and can therefore easily handle high levels of partnership concurrency.
Next, we implemented EPA and three instantaneous partnership approaches
in an existing model of heterosexual HIV transmission in Eswatini,
and examined differences in model outputs
under equal and approach-specific (recalibrated) parameters.
We found that model outputs were similar between EPA and
an instantaneous approach that allowed partnership change rates $\ll$ 1, whereas
two instantaneous approaches that forced partnership change rates to be $\ge$ 1
severely overestimated transmission via long-term partnerships,
even after recalibrating parameters, relative to EPA.
% TODO: not sure how to conclude
