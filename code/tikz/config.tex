\usepackage{tikz}
\usepackage{newtxsf}
\usepackage[tabular,osf,sfdefault]{gandhi}
\usepackage{microtype}\DisableLigatures{shape=sc}
\usepackage[italic]{mathastext}
\usepackage{bm,xfp}
% theme
\definecolor{red}{HTML}{CC0033}
\definecolor{blu}{HTML}{0099CC}
\definecolor{yel}{HTML}{FF9900}
\colorlet{C1}{red}\colorlet{C1a}{C1!50}
\colorlet{C2}{blu}\colorlet{C2a}{C2!50}
% structure
\colorlet{link}{red}
\colorlet{cite}{red!67!blu}
\colorlet{url}{blu}
% model: groups
\definecolor    {g}{rgb}{.60,.60,.60}
\definecolor  {all}{rgb}{.40,.00,.40}
\definecolor   {al}{rgb}{1.0,.40,1.0}
\definecolor   {am}{rgb}{.80,.20,.80}
\definecolor   {ah}{rgb}{.60,.00,.60}
\definecolor   {aq}{rgb}{.60,.20,.60}
\definecolor    {w}{rgb}{.80,.20,.20}
\definecolor   {wq}{rgb}{1.0,.60,.60}
\definecolor   {wl}{rgb}{1.0,.60,.60}
\definecolor   {wm}{rgb}{.80,.40,.40}
\definecolor   {wh}{rgb}{.60,.00,.00}
\definecolor{fsw.l}{rgb}{.60,.20,.20}
\definecolor{fsw.h}{rgb}{.40,.00,.00}
\definecolor  {fsw}{rgb}{.60,.00,.00}
\definecolor    {m}{rgb}{.20,.20,.80}
\definecolor   {mq}{rgb}{.60,.60,1.0}
\definecolor   {ml}{rgb}{.60,.60,1.0}
\definecolor   {mm}{rgb}{.40,.40,.80}
\definecolor   {mh}{rgb}{.00,.00,.60}
\definecolor{cli.l}{rgb}{.20,.20,.60}
\definecolor{cli.h}{rgb}{.00,.00,.40}
\definecolor  {cli}{rgb}{.00,.00,.60}
% model: health
\definecolor  {sus}{rgb}{1.0,.60,.00}
\definecolor  {hiv}{rgb}{.90,.00,.15}
\definecolor  {ahi}{rgb}{.90,.30,.00}
\definecolor{hiv.1}{rgb}{.90,.15,.00}
\definecolor{hiv.2}{rgb}{.90,.00,.15}
\definecolor{hiv.3}{rgb}{.60,.00,.30}
\definecolor {aids}{rgb}{.30,.00,.60}
% model: cascade
\definecolor  {udx}{rgb}{.00,.30,.30}
\definecolor   {dx}{rgb}{.00,.60,.30}
\definecolor   {ux}{rgb}{.00,.60,.60}
\definecolor   {tx}{rgb}{.40,.80,.00}
\definecolor   {vx}{rgb}{.80,.80,.00}
% model: partnerships
\definecolor  {msp}{rgb}{.26,.04,.41}
\definecolor  {cas}{rgb}{.58,.15,.40}
\definecolor  {swo}{rgb}{.87,.32,.23}
\definecolor  {swr}{rgb}{.99,.65,.04}
% foi
\definecolor{ird}{HTML}{00CC99} % green
\definecolor{iry}{HTML}{0099CC} % blue
\definecolor{ipy}{HTML}{CC00CC} % purple
\definecolor{epa}{HTML}{FF9900} % orange
\definecolor{ptc}{HTML}{FF0066} % pink
% art
\definecolor{caslo}{HTML}{0066CC} % blue
\definecolor{casmd}{HTML}{009999} % teal
\definecolor{cashi}{HTML}{00CC66} % green

\usetikzlibrary{
  decorations.pathreplacing,
  arrows.meta,
  plotmarks,
  fadings,
  fit}
\pgfdeclarelayer{back}
\pgfdeclarelayer{fore}
\pgfsetlayers{back,main,fore}
\tikzset{
  > = latex,
  lwd/.style = {
    line width = .2mm,
  },
  df/.style = {
    draw = #1,
    fill = #1!20,
    inner sep = 0pt,
    lwd,
  },
  state/.style = {
    df = #1,
    rectangle,
    minimum width = 8em,
    minimum height = 3em,
    rounded corners = .5em,
    anchor = mid,
  },
  arrow/.style = {
    line width = .5mm,
    draw = #1,
  },
  plot/.style 2 args = {
    thick,
    variable = \x,
    samples = 128,
    domain = #1:#2,
  },
  tangent/.style 2 args = {
    plot = {#1}{#2},
    samples = 2,
    mark = o,
    mark size = .5mm,
    shorten <= .5mm,
    shorten >= .5mm,
  },
  brace/.style = {
    decorate,
    decoration = {brace,amplitude=1ex},
  },
  axlab/.style = {
    align = center,
  },
  pair/.style = {
    df = #1,
    circle,
    text = #1,
    minimum width = 1.67em,
    minimum height = 1.67em,
  },
  unobs/.style = {#1!40},
  partx/.style = {very thick},
  parti/.style = {partx,{Circle[length=4pt]}- },
  parto/.style = {partx, -{Rays[length=6pt]}},
  partb/.style = {partx,{Circle[length=4pt]}-{Rays[length=6pt]}},
  partz/.style = {partx, -{Square[length=4pt]}},
}
% paths
\newcommand{\rootpath}{../..}
\graphicspath{{\rootpath/out/fig/tikz/}}
\newcommand{\ifempty}[3]{\if\relax\detokenize{#1}\relax#2\else#3\fi}
% scopes
\newcommand{\shiftscope}[3]{%
  \begin{scope}[transform canvas={xshift=#1,yshift=#2}] #3 \end{scope}}
\newcommand{\clipscope}[5]{
  \scope\pgfinterruptboundingbox
    \clip (#1,#2) rectangle (#3,#4);
  \endpgfinterruptboundingbox #5 \endscope}
% binom
\newcommand{\binomaxes}{
  \draw[->] (0,0) -- (0,1.1);
  \draw[->] (0,0) -- (110,0);
  \ifbinomfull \draw[dashed] (0,1) node[left,axlab]{1} -- (110,1); \fi}
\newcommand{\binomlabs}{
  \node[axlab,below] at (50,0) {$A$\ifbinomfull: number of sex acts\fi};
  \node[axlab,above,rotate=90] at (0,.5)
    {$B(A)$\ifbinomfull: probability\\of transmission\\per partnership\\\fi};
  \node[axlab,below left] at (0,0) {0};}
\newcommand{\slice}[5]{
  \ifx#3#4\draw[draw=#5,fill=#5!20] ({#1},{#2}) circle (2mm);
  \else\draw[draw=#5,fill=#5!20] ({#1},{#2}) --
    ++ ({#3*360+90}:2mm) arc ({#3*360+90}:{#4*360+90}:2mm) -- cycle; \fi}
% partnerships
\newif\ifdrawpairs
\newcommand{\defrecall}[2][0]{\def\xri{#1}\def\xro{#1+#2}}
\newcommand{\partnership}[4]{% xi,dx,y,style
  \def\xpi{#1+.01}\def\xpo{#1+#2-.01}
  \ifnum1=\fpeval{(\xpi>\xro)|(\xpo<\xri)} % fully unobserved
    \draw[partb,#4,unobs=gray] (\xpi,#3) -- (\xpo,#3);
  \else\ifnum1=\fpeval{\xpo>\xro} % right censored
    \draw[parti,#4] (\xpi,#3) -- (\xro,#3);
    \draw[parto,unobs=#4] (\xro,#3) -- (\xpo,#3);
  \else % fully observedd
    \draw[partb,#4] (\xpi,#3) -- (\xpo,#3);
  \fi\fi}